\chapter{Texts}\label{texts}\exyewidth{(55)}

\newcommand{\res}{}
\newcommand{\lnt}{\newline Laves' note: }
\newcommand{\llg}{\newline Laves text difference: }
\newcommand{\np}{}
\newcommand{\ph}{}
\newcommand{\nq}{Q: }
\renewcommand{\nt}{\newline Note: }
\newcommand{\fv}[1]{\mor{#1}}
\renewcommand{\uex}[1]{\textit{#1}\vspace*{1.5pt}}

Here are a few texts in Bardi and Jawi. The texts were chosen to illustrate different periods of the Bardi record and different dialects of the language. The texts also illustrate various subjects from traditional Bardi narratives. Three stories are included from the Laves materials. Text 130 is in Jawi (see \sref{s:jawi}). It shows the characteristic third person past augment prefix combination \mor{nyarr-}. There are also a number of phonetic differences between the Laves materials and Modern Bardi, some of which are probably due to researcher transcription differences, while others are probably genuine differences in the language. Work on these differences is ongoing.

These texts were glossed in Toolbox (\texttt{\small www.sil.org/computing/toolbox}). The examples in the rest of the grammar were glossed manually, using text expansion macros. Vowel length has not been marked on words in the texts.

\section{Laves}
Gerhardt Laves spent approximately four months working with Bardi speakers in 1928--1929, on Sunday Island, Boolgin, and Lombadina (with the majority of the time spent on Sunday Island). Laves' texts are preceded by a r\'esum\'e, a summary of the story in English which Laves seems to have collected before getting the story in Bardi. I have worked through the Laves materials with several Bardi speakers in 2003 and 2008. Laves annotated his own texts. I have included some explanatory annotations in these texts, where the language recorded is very different from the modern language. Where there is a note of the form Laves text difference: X : Y [annotation], this signals a substantive amendment to the original text. The first item is the correction, the second is the original form, and the annotation provides a comment on what speakers said about the original form and why it was changed. Spelling differences between Laves and the modern language are silently corrected (so that words can be easily looked up in the Bardi dictionary), but final vowels are left as in the original.

\subsection{Text 142: Goolamana}\label{Goolamana}
This story was told to Laves by Agoomoo (page reference: 14/3961-3 (text); 3964-66 (res)). The story concerns the Dreamtime figure Goolamana, who crossed over from Sunday Island to the mainland on the eastern shore of King Sound. There he met a man speaking a Worrorran language, and since they did not speak each other's languages, they both turned to stone.

\setcounter{exxy}{0}\begin{exye}
\exy \uex{Goolamana barda   jarrgany   inamana                                 gaalwangan.}
\gll Goolamana barda   jarrgany   i-na-ma-na            gaalwa-ngan.\\
G.        away    cut.across \tsc{3m}-\tsc{tr}-put-\tsc{rem.pst} raft-\tsc{all}\\
\glt Goolamana crossed over to his raft.

\exy \uex{Daab     innyana                                 garrin.}
\gll Daab     i-n-nya-na            garrin.\\
climb.up \tsc{3m}-\tsc{tr}-get-\tsc{rem.pst} hill\\
\glt  He climbed a hill.

\exy \uex{Booroo inangoorroon                             jarrara nirirrjarra    gaalwanyarra   barda   inanggalananana.}
\gll Booroo i-na-ngoorroo-n          jarrar  nirirrjarra         gaalwa-nyarr  barda   i-na-ng-gala-na-na.\\
place  \tsc{3m}-\tsc{tr}-stick.to-\tsc{cont} harpoon go.along.coast raft-\tsc{comit}   off     \tsc{3m}-\tsc{tr}-\tsc{pst}-move-\tsc{cont}-\tsc{rem.pst}\\
\glt  He paddled/pushed along with his raft along the shore, with his turtle spear.

\exy \uex{Nyoonoogiji     inamanirr                                  gaalwa gala barda   daab      innyana                                 alalon           booroo garrirn.}
\gll Nyoonoo=gij    i-na-ma-n=irr  gaalwa gala barda   daab      i-n-nya-na            alal-on      booroo garrin.\\
there=very   \tsc{3m}-\tsc{tr}-put-\tsc{cont}=\tsc{3a.DO}  raft   now  away    go.ashore \tsc{3m}-\tsc{tr}-get-\tsc{rem.pst} steep-\tsc{loc} place  hill\\
\glt  Here he put his raft [on the shore] and climbed up along the edge up a hill.

\newpage\exy \uex{Injalana                                Goolamananimi        ginyinggiamba         arrangala ginyinggi  aamba niinga Gaminjina Goolamananimi        injayboon                                    Gaminjin.}
\gll I-n-jala-na            Goolamana-nim       ginyinggi=amb       arrangala ginyinggi  aamba niinga  Gaminjina Goolamana-nim       i-n-jayboo-n          Gaminjina.\\
\tsc{3m}-\tsc{tr}-see-\tsc{rem.pst} G.-\tsc{erg} \tsc{3min}=\tsc{rel} bush        \tsc{3min} man   name   G.        G.-\tsc{erg} \tsc{3m}-\tsc{tr}-ask.question-\tsc{cont} G.\\
\glt  Goolamana saw a man, by the name of Gaminjin; Goolamana asked Gaminjin a question.

\exy \uex{Goolamananimi        aranga ngaanka  injayboona                                      arra      oolarlamankana                                  Gaminjinanim         rooban     barna injoonina                                             Gaminjinanimi        aranga ngaanka  Goolamananim         arra      nimoonggoonjini                 ngaanka.}
\gll Goolamana-nim       arang  ngaanka  i-n-jayboo-na            arra      oo-la-lamanka-na            Gaminjina-nim       rooban     barn  i-n-joo-ni-na            Gaminjina-nimi       aranga  ngaanka  Goolamana-nim       arra      ni-moonggoon=jini         ngaanka.\\
G.-\tsc{erg} other  language \tsc{3m}-\tsc{tr}-ask.question-\tsc{rem.pst} \tsc{neg} \tsc{3m.irr}-\tsc{irr}-hear-\tsc{rem.pst} G.-\tsc{erg} in.return  tell  \tsc{3m}-\tsc{tr}-do/say-\tsc{cont}-\tsc{rem.pst} G.-\tsc{erg} other  language G.-\tsc{erg} \tsc{neg} \tsc{3m}-know=\tsc{3m.IO} language\\
\glt  Goolamana used a different language, Gaminjin didn't hear/understand him, and spoke back in a different language, which Goolamana didn't know.%\enlargethispage*{1cm}

\exy \uex{Ginyinggon ginyinggi  Gaminjini       goolboo injoonoo                                  ginyinggi  aamba arrangalba    aamba.}
\gll Ginyinggon ginyinggi  Gaminjina goolboo i-n-joo-na            ginyinggi  aamba arrangal=ba  aamba.\\
then       \tsc{3min} G.  rock    \tsc{3m}-\tsc{tr}-become-\tsc{rem.pst} \tsc{3min} man   bush man\\
\glt  Then, this Gaminjin turned into a rock, he was a bush man.

\newpage\exy \uex{Goolamanajina             ngaanka  Oowini arra      nimoonggoon         ginyinggi  Gaminjinanim.}
\gll Goolamana=jina           ngaanka  Oowini arra      ni-moonggoon ginyinggi  Gaminjina-nim.\\
G.-\tsc{3m.poss} language Oo.    \tsc{neg} \tsc{3m}-know      \tsc{3min} G.-\tsc{erg}\\
\glt  Goolamana's language is Oowini, and Gaminjin didn't know it.

\exy \uex{Goolamana ginyinggon goolboo injoonoo.}
\gll Goolamana ginyinggon goolboo i-n-joo-noo.\\
G.        then       stone   \tsc{3m}-\tsc{tr}-become-\tsc{rem.pst}\\
\glt  Goolamana turned to stone.

\exy \uex{Imbanya.}
\gll i-m-banya.\\
\tsc{3m}-\tsc{pst}-finish\\
\glt  He died.
\end{exye}

\subsection{Text 75: Story about Mirrdiidi people}

Laves did not make a note of who told this story to him. (The page reference in the Laves corpus is 11/3297.)

Bessie Ejai gave the following explanation of fairy creatures called variously \mor{Mirrdidi},  \mor{Mirrdiidi}, or \mor{Mirridid(i)}.  \mor{Mirrididi} were fairy kids who lived all through the Dampier Peninsula and the Buccaneer Archipelago. When a stranger came to a new place, the \mor{Mirrididi} would come up to him
at night asking for their dad. Sometimes they would come in the shape of a kangaroo. If someone killed a \mor{Mirridid} kangaroo during the day, at night in the camp the kangaroo would change back to a \mor{Mirridid} person and say `now I'll call you dad, because you speared me.' They mostly came to men, but
they could come to women too. They are a little like \mor{raya} spirits, except \mor{raya} can still be felt and seen on the
islands, but the \mor{Mirrididi} have all gone now. They are also a little like \mor{ingarda}, the spirits who gave \mor{ilma} to men. They used to make \mor{giidoonoo}, or little houses.

Laves provided the following summary of the text: \newline \mor{Mirrdiidi} [\mor{mirridid}] were little people from a long time ago. They fought people. They made a little house (\mor{giidoonoo}). Somebody
fought them and another man (a \mor{boordij} `big' fellow). The \mor{Mirrdiidi} went inside their little house. At night time (i\mor{idanngoorroo}) another man came
up and made a fire and burned up the Mirrdiidi in the \mor{giidoonoo}. The Mirrdiidi could not get out before dying. The big man ran away and nobody caught him. This is the end of the story in which the little fellow had been so pugnacious.\setcounter{exxy}{0}

\begin{exye}
\exy \uex{``Barda arr ngandanngay.''}
\gll ``Barda arr nga-n-da-n=ngay.''\\
off go \tsc{trans.imp}-\tsc{tr}-do/say-\tsc{cont}=\tsc{1m.DO}\\
\ft ``Look, I'm going.''
%\newpage
\exy \uex{Indanjina. ``Baanigarra jiy balaboo?''}
\gll I-n-da-n=jina. ``Baanigarr jiya balaboo?''\\
\tsc{3m}-\tsc{tr}-do/say-\tsc{cont}=\tsc{3m.IO} when \tsc{2m.poss} here\\
\ft ``He says to him. When are you coming?''
\llg \mor{jiy: joo} [case difference? idiom difference? error?]
\exy \uex{``Barnanggarra ngay barda. Nyalaboo dirrayi ngonkonji barnanggarra.''}
\gll ``Barnanggarr ngay barda. Nyalaboo dirray ngo-n-k-onji barnanggarra.''\\
today \tsc{1min} off over.there turn.around \tsc{1m}-\tsc{tr}-\tsc{fut.tr}-share now\\
\ft ``I'm going today. I'll come back today today.''
\nt More idiomatic would be \mor{Barnanggarra ngayoo barda arr ngandan.}
\llg \mor{dirrayi ngonkonji: dirr inganggornji} [this is unparsable; it is probably a garbling of two separate phrases]
\exy \uex{``Ngarri ngay injilnganjan goolan.''}
\gll ``Ngarri ngay i-n-jilngi-n=jan gool=an.''\\
a.lot \tsc{1min} \tsc{3m}-\tsc{tr}-tell-\tsc{cont}=\tsc{1m.IO} father=\tsc{1m.poss}\\
\ft ``{\dots} My father is telling me all the time:''
\llg \mor{ngarri: mirri} [not known; not a Bardi word, so we changed it to one that made some sense] \\lg \mor{injilnganjan: injoolnganjan} [variant choice
difference?; variation in vowel harmony of root?] 
\nt \mor{goolan}: I assume that this comes from \mor{gooloo-jan} `my father' with contraction, and is
not the word \mor{goolan} `bluebone'. \mor{Gooloonimjan} would be an acceptable alternative here in the modern language.%\enlargethispage*{.6cm}
\exy \uex{``Joornkoojoornk arr ngandan marrany yoorr [inaman] gaarroogoon, nyoonoo inboojan aarli gaarragoon.}
\gll ``Joornkoo-joornk arr nga-n-d-an marrany yoorr [i-na-ma-n] gaarroo-goon, nyoonoo i-n-booja-n aarli gaarra-goon.\\
Quickly-\tsc{redup} go \tsc{1m}-\tsc{tr}-do/say-\tsc{cont} quickness come.down \tsc{3m}-\tsc{tr}-put-\tsc{cont} sea-\tsc{loc} there \tsc{3m}-\tsc{tr}-spear-\tsc{cont} fish sea-\tsc{loc}\\
\ft ``I'm running along the road, and he's spearing my fish in the sea.''
\nt The word \mor{inaman} was added because \mor{yoorr} `come down' can't be used without it in Modern Bardi.
\llg \mor{Joornkoojoornk arr: Joornkoojoornkarra} [word boundary division; arra \> arr]
\llg \mor{gaarragoon: gaarroogoon} [vowel harmony difference]
\exy \uex{``Daab innyan aarlinyarra. Inmarrarnirr noorroogoon.''}
\gll ``Daab i-n-nya-na aarli-nyarr. i-n-marra-rn=irr noorroo-goon.''\\
climb.up \tsc{3m}-\tsc{tr}-catch-\tsc{rem.pst} fish-\tsc{comit} \tsc{3m}-\tsc{tr}-cook-\tsc{cont}=\tsc{3a.DO} fire-\tsc{loc}\\
\ft ``He climbed up with the fish. He cooked [the fish] on the fire.''
\nt The comitative marker frequently has the form \mor{-nyarra} in the Laves texts, but it's always \mor{-nyarr} in the modern language.
\llg \mor{Inmarrarnirr: inmarrarn} [singular for plural; plural better here]
\exy \uex{``Inarnirr aranga. Aranga nganjalgardirr arra ngalanirr.''}
\gll I-n-ar-n=irr arang. arang nga-n-jalg=ard=irr arra nga-la-n=irr.''\\
\tsc{3m}-\tsc{tr}-pierce-\tsc{rem.pst}=\tsc{3a.DO} others others \tsc{1m}-\tsc{tr}-hide=maybe=\tsc{3a.DO} \tsc{neg} \tsc{1m}-\tsc{irr}-[give]-\tsc{cont}=\tsc{3a.DO}\\
\ft ``He speared other fish. I hid the others and didn't give them [to him].''
\newpage\exy \uex{``Booroo nganjina arinyji aranga Mirridiidi ingarrangoorroobi niiman aamba.''}
\gll ``Booroo nga-n-ji-na arinyji arang Mirridid i-ng-arr-a-ngoorribi niimana aamba.''\\
look.around \tsc{1m}-\tsc{tr}-do/say-\tsc{rem.pst} one other little.people \tsc{3m}-\tsc{pst}-\tsc{aug}-\tsc{tr}-chase.after too many man\\
\ft ``I looked around and many of the \mor{Mirridiidi} little people were chasing him.''
\llg \mor{Mirridiidi: Mirrdidi} [pronunciation difference? transcription difference?]
\exy \uex{Arra oolirrinyana (giinyji) inyjiibi boordiji fella aamba.}
\gll Arra oo-la-rr-inya-na (giinyji) i-ny-jiibi boordij fellow aamba.\\
\tsc{neg} \tsc{3m.irr}-\tsc{irr}-\tsc{aug}-catch-\tsc{rem.pst} without.speaking \tsc{3m}-\tsc{pst}-die big { } man\\
\ft They didn't catch the guy, who died without saying anything.
\nt The word \mor{giinyji} here appears to be a misparsing if it's related in any way to \mor{irrmarlginyji} `they kept themselves quiet'. It's not a Modern Bardi word in this meaning; the word these days refers to a blockage or obstruction.
\exy \uex{Inamalanirr gooljoongoo noorroo. Ingirrjiimbina boonyjangarr.}
\gll I-na-mala-n=irr gooljoo-ngoo noorroo. i-ng-irr-jiimbi-na boonyj=angarr.\\
\tsc{3m}-\tsc{tr}-burn-\tsc{cont}=\tsc{3a.DO} grass-\tsc{ins} fire \tsc{3m}-\tsc{pst}-\tsc{aug}-die-\tsc{rem.pst} everyone=\tsc{only}\\
\ft He put the grass on the fire. They all died.
\nt A more natural way to say \mor{ingirrjimbina boonyjangarr} these days would be \mor{boonyjamb ingirrjimbina}, making the causal connection a bit clearer.
\exy \uex{Joornk innyamba boordij aamba. Ara Mirridiidi inoongoorribin arinyjinim. Arra oolinyana arra oolona.}
\gll Joornk i-n-ny=amba boordij aamba. Ara Mirridiidi i-noo-ngoorribi-n arinyji-nim. Arra oo-l-inya-na arra oo-l-o-na.\\
speed \tsc{3m}-\tsc{tr}-catch=\tsc{rel} big man one little.people \tsc{3m}-\tsc{tr}-chase.after-\tsc{cont} one-\tsc{erg} \tsc{neg} \tsc{3m.irr}-\tsc{irr}-catch-\tsc{rem.pst} \tsc{neg} \tsc{3m.irr}-\tsc{irr}-spear-\tsc{rem.pst}\\
\ft The large man ran away. One of the other Mirridiidi men chased him, but he never caught him or speared him.
\llg \mor{inoongoorribin: inangarribina} [pronunciation difference]
\exy \uex{Niimarlga inyjiibina iiga danarnina inyjiibina.''}
\gll Niimarlga i-ny-jiibi-na iiga darr i-n-ar-n=jin i-ny-jiibi-na.''\\
By.himself \tsc{3m}-\tsc{pst}-die-\tsc{rem.pst} sickness refuse \tsc{3m}-\tsc{tr}-pierce-\tsc{cont}=\tsc{3m.IO} \tsc{3m}-\tsc{pst}-die-\tsc{rem.pst}\\
\ft He died by himself; he got sick and died.
\end{exye}

\subsection{Text 130: Jawi}
This text is in the Jawi language variety, told to Laves by Jamboo (page references are 14/3821-7 (text); 3827-31 (res)).  Note that in this text there are a number of words which were not known. This is the story of a tidal wave that inundated Sunday Island. The timing of the wave is unknown, but given the proximity of the northwest Australian coast to Indonesia, stories of tidal waves are not surprising. The tidal waves resulting from the eruption of Krakatoa in 1883 would have affected the area, and it is possible that this story refers to that event.

Laves' summary (abridged): That dry (season) they were all together outside. This was a long time ago. The tide went out really far, so that the reef was uncovered, and the channel as far as the mainland. Then the water came up as far as the Sunday Island graveyard. Only that mountain Goorililiwa, that one rock, was above water. All the people went altogether to that one hill.

\setcounter{exxy}{0}\begin{exye}
\exy \uex{Inyjoordina  gaarra boonyja.}
\gll i-ny-joordi-na gaarra boonyja.\\
\tsc{3m}-\tsc{pst}-dry.up-\tsc{rem.pst} salt.water all\\
\ft The sea all dried up.
\nt That is, the tide went out in advance of a tsunami.
\exy \uex{Niyangala oola gaarra gaanyga aamba, oorany baawa barda nyoonbirroonan nyoonoonya biindanola ``nyalamboo anarra'' arinyji garrin nyarrawarn arinyji garrin.}
\gll Niyangala oola gaarra gaanyga aamba, oorany baawa barda nyanbirroonony nyoon-oonyi biindan-ola ``nyalab anarra'' arinyji garrin nyarra-warn arinyji garrin.\\
tongue water salt.water mainland man woman child off other.side out.there-\tsc{lat} bush-origin over.there pick.up? one hill 3.\tsc{pst.aug-}climb? one hill\\
\ft There were little tongues of water and the men, women and children from the `other side' came this side-they picked up the kids and went up a hill.
\nt The form \mor{nyarrawan} is a particularly Jawi verb form. It is not clear what the verb root is, and this root does not occur in mainland Bardi.
\nt \mor{bindonola} may refer to `mainland Oowini' (per BE and NI)
%\nt CLB note: nyalamboo anarra ``pick them up''
%\nt note on bindonola. Vowel harmony and non-direction term for-ola.
\exy \uex{Nyalamboo ambooriny arinyji garrin, inyjoordina  booroo boonyja ambooriny arinyji garrin.}
\gll Nyalamboo ambooriny arinyji garrin i-ny-joordi-na booroo boonyja ambooriny arinyji garrin.\\
over.there people one hill \tsc{3m}-\tsc{pst}-dry.up-\tsc{rem.pst} place all people one hill\\
\ft The people were all on one hill; as the tide went out everyone was on one a single hill.
\nt I assume that \mor{nyalamboo} is the same as Bardi \mor{nyalab} `over there'.
\exy \uex{Arinyjoon booroo imbarnarnina  boonyj imbanyana  ambooriny arinyji garrin imbarnarna  arinyji garrin.}
\gll Arinyj-oon booroo i-m-barna-ni-na boonyja i-m-banya-na ambooriny arinyji garrin i-m-barna-na arinyji garrin.\\
one-\tsc{loc} place \tsc{3m}-\tsc{pst}-sit.down.together-\tsc{cont}-\tsc{rem.pst} all \tsc{3m}-\tsc{pst}-finish-\tsc{rem.pst} person one hill \tsc{3m}-\tsc{pst}-sit.down.together-\tsc{rem.pst} the.very.one hill\\
\ft People from the same area sat down together; they congregated on a single hill.
%\nt CLB: imbarnarnina asterisked with imbarnanij 'sit down together'
\exy \uex{Ninga ginyinggi garrin Goorililiwa.}
\gll Ninga ginyinggi garrin G.\\
name \tsc{3min} hill G.\\
\ft The name of this hill is Goorililiwa.
%\nt Laves wrote niinga
%\nt Goorililiwa asterisked with ``Not on S.I. mainland''
\exy \uex{Inyjoordina  imbanyana  garra ginyinggarra Iwanyoo injalgoordoo gaa-rranim nyoonoo gaanyga injalgoord boonyja imboolana  gaarra booroo.}%\enlargethispage*{.7cm}
\gll i-ny-joordi-na i-m-banya-na garra ginyinggarra Iwanyoo i-n-jalgoordoo gaarra-nim nyoonoo gaanyga i-n-jalgoord boonyja i-m-boola-na gaarra booroo.\\
\tsc{3m}-\tsc{pst}-dry.up-\tsc{rem.pst} \tsc{3m}-\tsc{pst}-finish-\tsc{rem.pst} on.and.on and.then Sunday.Island \tsc{3m}-\tsc{tr}-cover.up salt.water-\tsc{erg} there mainland \tsc{3m}-\tsc{tr}-cover.up all \tsc{3m}-\tsc{pst}-come.out-\tsc{rem.pst} salt.water place\\
\ft The tide went right out. And then the sea covered Sunday Island, and covered up all the mainland, and the sea came right in over the land.
\nt The name for Sunday Island is \mor{Iwany} or \mor{Iwanyi} in Bardi, but usually appears as \mor{Iwanyoo} in Jawi materials.
\exy \uex{Iwanyoo inggardina gala jimbin.}
\gll Iwanyoo i-ng-gardi-na gala jimbin.\\
Sunday.Island \tsc{3m}-\tsc{pst}-drown-\tsc{rem.pst} now underneath\\
\ft Sunday Island went right under.
\exy \uex{Daab innyoonoo Mayalan---imboolana  nyoonoo Goorililiwa imbarnarnin.}
\gll Daab i-n-nya-na Mayal-an i-m-booloo-na nyoonoo G. i-m-barna-n-in.\\
climb.up \tsc{3m}-\tsc{tr}-catch-\tsc{rem.pst} Mayala-\tsc{all} \tsc{3m}-\tsc{pst}-come-\tsc{rem.pst} there G. \tsc{3m}-\tsc{pst}-sit.down.together-\tsc{cont}-\tsc{rem.pst}\\
\ft It [the water] climbed up to Mayala--it came to the place where they were all together.
\exy \uex{Imbanyana  boordij alalgoordoo inyjarrmina jamarda boordij alalgoordoo inggarndandarrara jamarda gooyarra ingarrgandandarrar alalgoordoo boordij alalgoordoo nyoonoo garrin ginyinggi.}
\gll I-m-banyi-na boordij alalgoordoo i-ny-jarrmi-na jamarda boordij alalgoordoo i-ng-garndarrma jamarda gooyarra i-ng-arr-garndarrma alalgoord boordij alalgoord nyoonoo garrin ginyinggi.\\
\tsc{3m}-\tsc{pst}-finish-\tsc{rem.pst} big wave \tsc{3m}-\tsc{pst}-rise-\tsc{rem.pst} this.way big wave \tsc{3m}-\tsc{pst}-swell.up coming.this.way two \tsc{3m}-\tsc{pst}-\tsc{aug}-swell.up wave big wave there hill \tsc{3min}\\
\ft The big wave came up, it came this way; two waves swelled up towards that hill.
\nt The Jawi verb here is \mor{inggarndandarrara}, but this verb is recorded as \mor{inggarndarrma} in Bardi.
\exy \uex{Arra oolajarrmana,  garda dalboon.}
\gll Arra oo-la-rr-jarrma-na, garda dalboon.\\
\tsc{neg} \tsc{3m.irr}-\tsc{irr}-\tsc{aug}-rise-\tsc{rem.pst} still dry\\
\ft But they didn't go under, they stayed dry.
%\nt oolajarrmina NI corrected this verb to. (Also next example)
\exy \uex{Arra oolajarrmina  gaarra, inyjoordina  booroo gaarra, yoorr nyarrina  barda ambooriny jimbinngan joorrboo nyarroorn.}
\gll Arra oo-la-rr-jarrmi-na gaarra, i-ny-joordi-na booroo gaarra, yoorr nyarr-i-na barda ambooriny jimbin-ngan joorrboo nyarr-oo-rn.\\
\tsc{neg} \tsc{3m.irr}-\tsc{irr}-\tsc{aug}-rise-\tsc{rem.pst} salt.water \tsc{3m}-\tsc{pst}-dry.up-\tsc{rem.pst} place salt.water come.down \tsc{3m}-\tsc{pst}-\tsc{aug}-do/say-\tsc{rem.pst} away people underneath-\tsc{all} jump \tsc{3m}-\tsc{pst}-\tsc{aug}-do/say-\tsc{rem.pst}\\
\ft They didn't go into the sea, and the land dried out, and the people went down; they skipped down inside.
\llg \mor{nyarrina}: Bardi \mor{ingarraman} [dialectal difference]
%\nt war.l (Laves has waral) for walk-not retroflex l.
%\nt nyarrinan: Laves has nyaninan.
\nt \mor{nyarrina} is glossed by Laves as meaning  `stay in place'%\enlargethispage*{.7cm}
%\nt iwanyun (Bardi)
%\nt nyarralnana (Laves): walk around, nyarralanan
\newpage\exy \uex{Ginyinggarra barda warl nyarroorn  gaanyga jirri booroo warl nyarr-oorn  jamarda Iwanyoo jirra booroo.}
\gll Ginyinggarra barda warl nyarr-oo-rn gaanyga jirri booroo warl nyarr-oo-rn jamard Iwany jirra booroo.\\
and.then off come.for 3.\tsc{aug.pst}-do-\tsc{rem.pst} mainland \tsc{3a.poss} place come.for 3.\tsc{aug.pst}-do-\tsc{rem.pst} this.way Sunday.Island \tsc{3a.poss} place\\
\ft Then they walked to their country on the mainland, and they walked to their country on Sunday Island.
\nt The word \mor{warl} (Laves has \mor{waral}) has a cluster of \mor{r} + \mor{l}; it is not a retroflex \mor{l}.
\exy \uex{Ginyinggarra nyarralnana  booroogoon jina goona Mayalanmardan garda warl nyarroorn,  ginyinggarra nyarrinan  booroogoonjirr.}
\gll Ginyinggarra nyarr-al-na-na booroo-goon jina goona Mayal-an-mardan gardi wal nyarr-oo-rn, ginyinggarra nyarr-i-na-n booroo-goon=jirr.\\
and.then 3.\tsc{aug.pst}-move-\tsc{rem.pst} place-\tsc{loc} \tsc{3m.poss} back Mayala-\tsc{all}-directive still come.for 3.\tsc{aug.pst}-do-\tsc{rem.pst} and.then 3.\tsc{aug.pst}-do-\tsc{rem.pst} call-\tsc{loc}=\tsc{3a.IO}\\
\ft Then they went to their camps in Mayala, then everyone was back in their own country.
\exy \uex{Ginyinggi imbanya gala inggoorrooma gala.}
\gll Ginyinggi i-m-banyi gala i-ng-oorrooma gala.\\
\tsc{3min} \tsc{3m}-\tsc{pst}-finish now \tsc{3m}-\tsc{pst}-?? that's.it\\
\ft That's the end, they all went.
\nt \mor{Inggoorrooma} is glossed as `bin all go' but the verb is not known from Bardi.
\end{exye}


\section{Modern stories}
The following stories are a small selection of those recorded as part of a Bardi oral history project in 2001 (August to December), along with a few recorded by Gedda Aklif\index{Aklif, Gedda} over the period 1990--1993. About 30 hours of narratives were recorded in Bardi and English as part of an oral history project, while Aklif recorded about 10 hours of narratives. 

\subsection{GAL1: The story of Galaloong}
This story was recorded from David Wiggan by Gedda Aklif in 1990. \index{Wiggan, David} It is a public version of the story of the Bardi culture hero Galaloong.

\setcounter{exxy}{0}\begin{exye}
\exy \uex{Jarri inanggalanan  Galaloong boonyja booroo: Nyoolnyool Baniyola, nyalabal boora nirirr injoonoo.}
\gll Jarri i-na-ng-gala-na-n Galaloong boonyja booroo Nyoolnyool Baniyol, nyalab-al booroo=a nirirrjarr i-n-joo-na.\\
this \tsc{3m}-\tsc{tr}-\tsc{pst}-move-\tsc{rem.pst}-\tsc{pres} G. everywhere place Nyulnyul Eastern.people from.this.side-\tsc{indef} country=\tsc{pred} along.the.edge \tsc{3m}-\tsc{tr}-do/say-\tsc{rem.pst}\\
\ft Galaloong has been everywhere, Nyulnyul country and Bardi country, he's been along the edge of everywhere.
\exy \uex{Booroo injoombarna  irrnga, inamana  irrnga booroo ginyingg aamba, irrngirrngi arrooloongan booroo barnanggarr.}
\gll Booroo i-n-joombar-na irr-nga, i-n-ma-na irr-nga booroo ginyinggi aamba, irrngirrngi a-rr-jooloong-an booroo barnanggarr.\\
place \tsc{3m}-\tsc{tr}-name.place-\tsc{rem.pst} \tsc{3a-}name \tsc{3m}-\tsc{tr}-put-\tsc{rem.pst} \tsc{3a-}name place \tsc{3min} man names 1-\tsc{aug}-collect-\tsc{cont} place now\\
\ft He's been counting and naming places; this man named the places, and we use those names now.
\exy \uex{Nyalab jarr goolarr injoonoo.}
\gll Nyalab jarri goolarr i-n-joo-na.\\
over.there this west \tsc{3m}-\tsc{tr}-do/say-\tsc{rem.pst}\\
\ft He's been on the western side.
\newpage\exy \uex{Barnoorarra nyalab, jarri Ardiyooloon injoonoo,  jamb biila injoonoo  barda goolarr.}
\gll Barnoorarra nyalab jarri Ardiyooloon i-n-joo-na, jamb biila i-n-joo-na barda goolarr.\\
northerners over.there this One.Arm.Point \tsc{3m}-\tsc{tr}-do/say-\tsc{rem.pst} \tsc{when} also \tsc{3m}-\tsc{tr}-do/say-\tsc{rem.pst} away westwards\\
\ft He's been to Ardiyooloon and Barnoorarra (Willie Point to Cygnet Bay), and he went again to the west.
\exy \uex{Joowoono Gardiny jarri Gardiny jarri nyalabal boora inanggalanirr  ambooriny.}
\gll Joowoon-o Gardiny jarri Gardiny jarri nyalab=al boora i-na-ng-gala-n=irr ambooriny.\\
Swan.Point-\tsc{abl} Swan.Island this Swan.Island this over.there=\tsc{indef} place \tsc{3m}-\tsc{tr}-\tsc{pst}-visit-\tsc{cont}=\tsc{3a.DO} people\\
\ft From Joowoon he went to Swan Point and Swan Island, he went everywhere visiting the people.
\exy \uex{Inanirr  aarlimay bardagayoon.}
\gll I-na-n=irr aarlimay bardaga-yoon.\\
\tsc{3m}-\tsc{tr}-\tsc{pres}=\tsc{3a.DO} food tree-\tsc{source}\\
\ft He gave them food from the trees.
\exy \uex{``Jarri aarlimay joogarr, jarri boowa, arra goolarrarli, arralalanirr  gorna,'' injoonoojirr.}
\gll ``Jarri aarlimay joogarra, jarri boowa arra goo-la-arr-rli, a-arr-jalala-n=irr gorna,'' i-n-joo-na=jirr.\\
this food \tsc{2a.poss} this inedible.food \tsc{neg} \tsc{2a}-\tsc{irr}-\tsc{aug}-eat 1-\tsc{aug}-look.after-\tsc{cont}=\tsc{3a.DO} well \tsc{3m}-\tsc{tr}-do/say-\tsc{rem.pst}=\tsc{3a.IO}\\
\ft ``This food is yours, this one is rubbish food: don't eat it, look after them well,'' he told them.
\exy \uex{Ginyinggo inyjiidin iniinalang, inanggalanirr  ambooriny boonyja.}
\gll Ginyinggo i-ny-jiidi-n iniinalang, i-na-ng-gala-n=irr ambooriny boonyja.\\
from.there \tsc{3m}-\tsc{pst}-go-\tsc{cont} islands \tsc{3m}-\tsc{tr}-\tsc{pst}-visit-\tsc{cont}=\tsc{3a.DO} people everyone\\
\ft Then he went to the islands, and visited all the people.
\exy \uex{``Ginyinggamba joogarra ngaanka Bardi.''}
\gll ``Ginyinggamba joogarra ngaanka Bardi.''\\
that's.why \tsc{2a.poss} language Bardi\\
\ft ``This is your language, Bardi,'' (he told them).
\exy \uex{Ginyinggon barnin injoonoojirr,  ``barnin arraman  barni morr. Oola joogarra jiibi inin, biidini joogarra oola niiminiimi, joogarra oola,'' injoonoojirr.}
\gll Ginyinggon barn i-n-joo-na=jirr, ``barn a-rr-a-ma-n barni morr. Oola joogarra jiibi i-ni-n, biidin joogarra oola niiminiimi, joogarra oola,'' i-n-joo-na=jirr.\\
then tell \tsc{3m}-\tsc{tr}-do/say-\tsc{rem.pst}=\tsc{3a.IO} use 1-\tsc{aug}-\tsc{tr}-put-\tsc{cont} this.(way) way water \tsc{2a.poss} this \tsc{3m}-be.located-\tsc{cont} fresh.water \tsc{2a.poss} water sand.soakage \tsc{2a.poss} water \tsc{3m}-\tsc{tr}-do/say-\tsc{rem.pst}=\tsc{3a.IO}\\
\ft Then he told them, ``Do things this way. Here are your water, your rockholes, your soakages in the ground and in the sand,'' he told them.%\enlargethispage*{1cm}
\exy \uex{Jarrgany barda inamana  Ardiyooloonko, roowilbarda innyana  gaa-rragoon garndi.}
\gll Jarrgany barda i-n-ma-na Ardiyooloon-go, roowil=barda i-n-nya-na gaarra-goon garndi.\\
cut.across away \tsc{3m}-\tsc{tr}-put-\tsc{rem.pst} One.Arm.Point-\tsc{abl} walk=maybe \tsc{3m}-\tsc{tr}-catch-\tsc{rem.pst} sea-\tsc{loc} top\\
\ft He might have crossed over from Ardiyooloon. He might have walked on top of the water.
%\newpage
\exy \uex{Jooloom injayboonirr  ambooriny ``Oola janaboora goorrinyan?''}
\gll Jooloom i-n-jayboo-n=irr ambooriny ``Oola janaboora goo-arr-inya-n?''\\
Middle.Island \tsc{3m}-\tsc{tr}-ask.question-\tsc{cont}=\tsc{3a.DO} people water where \tsc{2a}-\tsc{aug}-catch-\tsc{cont}\\
\ft At Jooloom he asked the people, ``Where do you get your water from?''
\exy \uex{``Arra arinyj arrin oolanyarr oolab inarnjarda,  arramin jarda oola irr-joordoonba  bardamba jarr arrin joobool arrin aanyjoo Ardiyooloonngan, boonamba arrinyan oola Inyjololon,'' ingirrinijin}
\gll ``Arra arinyj arr-joo-n oola-nyarr oola=b i-n-ar-n=jard, arra jarda oola i-arr-joordoo-n=b barda=jamb jarri arr-joo-n joobool arr-joo-n aanyjoo Ardiyooloon-ngan, boonoo=jamb a-rr-inya-n oola Inyjololon,''  i-ng-arr-joo-na=jin.\\
\tsc{neg} the.very.one \tsc{1aug}-do/say-\tsc{cont} rain-\tsc{comit} rain=\tsc{rel} \tsc{3m}-\tsc{tr}-pierce-\tsc{cont}=\tsc{1a.IO} \tsc{neg} \tsc{1a.poss} water \tsc{3}-\tsc{aug}-dry.up-\tsc{cont}=\tsc{rel} away=\tsc{rel} this \tsc{aug}-2m-\tsc{cont} swim \tsc{1aug}-do/say-\tsc{cont} back One.Arm.Point-\tsc{all} over.there=\tsc{rel} 1-\tsc{aug}-pick.up-\tsc{cont} water I. \tsc{3m}-\tsc{pst}-\tsc{aug}-do/say-\tsc{rem.pst}=\tsc{3m.IO}\\
\ft ``We get our water when it rains---we don't have water when it dries up. We swim to Ardiyooloon and get water from Inyjololon,'' they told him.
\exy \uex{Ginyinggo roowil innyana,  ``arralaliyangay!'' injoonoojirr.}%\enlargethispage*{.7cm}
\gll Ginyinggo roowil i-n-nya-na, ``a-arr-galala-a=ngay!'' i-n-joo-na=jirr.\\
from.there walk \tsc{3m}-\tsc{tr}-catch-\tsc{rem.pst} 1-\tsc{aug}-follow-\tsc{fut}=1sgO \tsc{3m}-\tsc{tr}-do/say-\tsc{rem.pst}=\tsc{3a.IO}\\
\ft Then he walked on and said ``You lot follow me!''
\exy \uex{Jool inarna  ara niimidi goolboon ara niimid biila jool inarna.}
\gll Jool i-n-ar-na ara niimidi goolboo-goon ara niimidi biila jool i-n-ar-na.\\
kneel.down \tsc{3m}-\tsc{tr}-pierce-\tsc{rem.pst} one knee rock-\tsc{loc} one knee also kneel.down \tsc{3m}-\tsc{tr}-pierce-\tsc{rem.pst}\\
\ft He knelt down with one knee on the rock, and then with the other knee.
\exy \uex{``Jarramb joogarrirr oola mangir arralabanirr,''  injoonoojirr.}\enlargethispage*{1cm}
\gll ``Jarri=jamb joogarra-irr oola mangir arr-n-laba-n=irr,'' i-n-joo-na=jirr.\\
this=\tsc{rel} \tsc{2a.poss}-\tsc{3a} water always \tsc{1aug}-\tsc{tr}-have-\tsc{cont}=\tsc{3a.DO} \tsc{3m}-\tsc{tr}-do/say-\tsc{rem.pst}=\tsc{3a.IO}\\
\ft ``Now you will always have water,'' he said.
\exy \uex{Nimarla, jard inarn  booroo, ininjamba nimarla niimbal ininjamb.}
\gll Niimarl, jard i-n-ar-na booroo, i-ni-n=jamb niimarl niimbal i-ni-n=jamb.\\
hand put.weight.on \tsc{3m}-\tsc{tr}-pierce-\tsc{rem.pst} place \tsc{3m}-sit-\tsc{cont}=\tsc{rel} hand foot \tsc{3m}-sit-\tsc{cont}=\tsc{rel}\\
\ft He put weight on the place with his hand, where his foot had been.
\exy \uex{Ginyingg aamba ingananangirr  Jooloomb, inamboon  aarli nilngga, garrin inarlin nyoonoo.}
\gll Ginyinggi aamba i-nga-ni-na-ng=irr Jooloom=b, i-na-ng-boo-na aarli nilngga, garrin i-na-rli-n nyoonoo.\\
\tsc{3min} man \tsc{3m}-\tsc{1m}-be.located-\tsc{rem.pst}-appl.=\tsc{3a.DO} Middle.Island=\tsc{rel} \tsc{3m}-\tsc{tr}-\tsc{pst}-poke-\tsc{rem.pst} fish red.snapper hill \tsc{3m}-\tsc{tr}-eat-\tsc{cont} there\\
\ft This man was with them on Jooloom, and he speared \mor{nilngga} fish (red snapper), and he ate it on the hill.
\exy \uex{Wangalanganim aamba arra oolarrarlin  ginyingg aarli, nyoongoorl-jin minyjin arrarlin.}
\gll Wangalang-nim aamba arra oo-la-rr-rli-n ginyinggi aarli, nyoongoorl-jin minyjin a-rr-rli-n.\\
young.man-\tsc{erg} man \tsc{neg} \tsc{3m.irr}-\tsc{irr}-\tsc{aug}-eat-\tsc{cont} \tsc{3min} fish old.person-\tsc{group} only 1-\tsc{aug}-eat-\tsc{cont}\\
\ft Young people don't eat this fish; only old people can eat it.
\exy \uex{Inyjiidin barda Jayirri, ool inarnirr,  Jayirri inganananirr.}
\gll i-ny-jiidi-n barda Jayirri, oola i-n-ar-na=irr, Jayirri i-nga-ni-na-n=irr.\\
\tsc{3m}-\tsc{pst}-go-\tsc{cont} off Jackson.Island water \tsc{3m}-\tsc{tr}-pierce-\tsc{rem.pst}=\tsc{3a.DO} Jackson.Island \tsc{3m}-\tsc{1m}-sit-\tsc{rem.pst}-\tsc{cont}=\tsc{3a}\\
\ft He went to Jayirri and gave them water, he stayed with them on Jayirri.
\exy \uex{``Jiibadan booroo arra goolarrali  riiwa jiib inin, arra goolarralayi  ambooriny noongoo, alig oonkin  anggabanim oonkalan.''}
\gll ``Jiiba booroo arra goo-la-rr-a-la-joo riiwa jiya=b i-ni-n, arra goo-la-rr-ala-yi ambooriny noongoo, alig oo-n-k-i-n anggaba-nim oo-n-k-gala-n.''\\
here place \tsc{neg} \tsc{2a}-\tsc{irr}-\tsc{aug}-\tsc{tr}-\tsc{irr}-\tsc{2m} hole 2m.poss=\tsc{rel} \tsc{3m}-sit-\tsc{cont} \tsc{neg} \tsc{2a}-\tsc{irr}-\tsc{aug}-move-simul? people stomach ache \tsc{3m.irr}-\tsc{tr}-\tsc{fut.tr}-do/say-\tsc{cont} who-\tsc{erg} \tsc{3m.irr}-\tsc{tr}-\tsc{fut.tr}-move-\tsc{cont}\\
\ft ``Don't you go to this place, there's a hole there, don't go there---people who go there will get a weak stomach.''
\exy \uex{``Ginyinggi riiwa jiib inin,'' injoonoojirr.}
\gll ``Ginyinggi riiwa jiiba i-ni-n,'' i-n-joo-na=jirr.\\
\tsc{3min} hole here \tsc{3m}-be.located-\tsc{cont} \tsc{3m}-\tsc{tr}-do/say-\tsc{rem.pst}=\tsc{3a.IO}\\
\ft ``The hole is there,'' he said.
\exy \uex{Nyoononyi Ngarrigoon, ginyinggo Jalan inanggalanirr,  oola inanirr.}
\gll Nyoonoo Ngarrigoon, ginyinggo Jalan i-na-ng-gala-n=irr, oola i-na-n=irr.\\
there Ng. from.there Tallon.Island \tsc{3m}-\tsc{tr}-\tsc{pst}-visit-\tsc{cont}=\tsc{3a.DO} water \tsc{3m}-\tsc{tr}-\tsc{cont}=\tsc{3a.DO}\\
\ft He went to Ngarrigoon on the other side, then he went to Jalan, he visited them and gave them water.
\exy \uex{``Jarramba inin joogarra oola,'' injoonoojirr.}
\gll ``Jarri=jamb i-ni-n joogarra oola,'' i-n-joo-na=jirr.\\
this=\tsc{rel} \tsc{3m}-be.located-\tsc{cont} \tsc{2a.poss} water \tsc{3m}-\tsc{tr}-do/say-\tsc{rem.pst}=\tsc{3a.IO}\\
\ft ``This is your water,'' he told them.
\exy \uex{Ginyinggo ngoonjoon ool ingarrananan  joodinygo inanggalboon  oola gardamb ingarranan  biidininyarr jirr oola.}
\gll Ginyinggo ngoonjoon oola i-ng-arr-ga-na-nan joodiny i-na-ng-galboo-n oola gardi=jamb i-ng-arr-ga-na-n biidin-nyarr jirr oola.\\
from.there since water \tsc{3m}-\tsc{pst}-\tsc{aug}-bring-\tsc{rem.pst}-times always \tsc{3m}-\tsc{tr}-\tsc{pst}-dig.out-\tsc{cont} water still=\tsc{rel} \tsc{3m}-\tsc{pst}-\tsc{aug}-bring-\tsc{rem.pst}-\tsc{cont} fresh.water-\tsc{comit} \tsc{3a.poss} water\\
\ft From that time they always had water. He dug for water and it's still there in the rock-holes.
\exy \uex{Ginyinggo goona Ralooraloo inananirr oola, riiwa inamoogarnajirr-nid  goolboondarr boonyja booroo bornkobornko jiiba goolboo.}
\gll Ginyinggo goona Ralooraloo i-n-ana-n=irr oola, riiwa i-na-moogar-na=jirr=nid goolboo-goondarr boonyja booroo bornkobornko jiiba goolboo.\\
from.there back R 3-\tsc{tr}-[give]-\tsc{cont}-\tsc{rem.pst}=\tsc{3a.DO} water hole \tsc{3m}-\tsc{tr}-make.something-\tsc{rem.pst}=\tsc{3a.IO}=\tsc{quant} rock-\tsc{loc_2} all place all.around here rock\\
\ft From there he went off back to Ralooraloo and gave them water: he made holes in the rock for them all around there.
\exy \uex{``Arra goolarralanirr  jarri miroorrambalirr,'' injoonoojirr.}
\gll ``Arra goo-la-rr-ala-n=irr jarri miroorroo=jamb-al=irr,'' i-n-joo-na=jirr.\\
\tsc{neg} \tsc{2a}-\tsc{irr}-\tsc{aug}-visit-\tsc{cont}=\tsc{3a} this powerful.place=\tsc{rel}-\tsc{indef}=\tsc{3a} \tsc{3m}-\tsc{tr}-do/say-\tsc{rem.pst}=\tsc{3a.IO}\\
\ft ``Don't go to this place---it's a sacred place for you,'' he told them.
\exy \uex{``Goorlil jorro jamb boor arralan, oondoordoo,'' injoonoojirr.}
\gll ``Goorlil jorro jamb boor arr-jala-n, oondoord,'' i-n-joo-na=jirr.\\
turtle from.here \tsc{when} watch.out.for \tsc{1aug}-look.at-\tsc{cont} married.turtle \tsc{3m}-\tsc{tr}-do/say-\tsc{rem.pst}=\tsc{3a.IO}\\
\ft ``Keep on looking out for married turtle from this place,'' he told them.
\exy \uex{``May jiiba irrngan joogarra, ngamari jiboordany joogarra,'' injoonoojirr.}\enlargethispage*{1cm}
\gll ``Mayi jiiba irrngan joogarra, ngamari jiboordany joogarra,'' i-n-joo-na=jirr.\\
plant.food here standing  \tsc{2a.poss} tobacco this.way \tsc{2a.poss} \tsc{3m}-\tsc{tr}-do/say-\tsc{rem.pst}=\tsc{3a.IO}\\
\ft ``This food grows for you, this tobacco is for you,'' he told them.
\newpage\exy \uex{Boolnginyi daab innyan  biila biligij gardoo.}
\gll Boolnginy-i daab i-n-nya-na biila biligij gardi.\\
Poolngin.Island-\tsc{loc} go.ashore \tsc{3m}-\tsc{tr}-catch-\tsc{rem.pst} also once.more yet\\
\ft Then he went ashore again at Boolnginy.
\exy \uex{Barn injoonoojirr,  ``jarramb joogarra baaliboor gardini nyoonoomba inin joogarra, ginyinggamba arralan, oonkoonkool darr oonkarn,''  injoonoojirr.}
\gll Barn i-n-joo-na=jirr, ``jarri=jamb joogarra baaliboor gardin nyoonoo=jamb i-ni-n joogarra, ginyinggamba a-rr-ala-n, oonkoonkool darr oo-n-k-ar-n,'' i-n-joo-na=jirr,\\
tell \tsc{3m}-\tsc{tr}-do/say-\tsc{rem.pst}=\tsc{3a.IO} this=\tsc{rel} \tsc{2a.poss} camp cave there=\tsc{rel} \tsc{3m}-be.located-\tsc{cont} \tsc{2a.poss} that's.why 1-\tsc{aug}-live-\tsc{cont} ocean.storm come \tsc{3m.irr}-\tsc{tr}-\tsc{fut.tr}-pierce-\tsc{cont} \tsc{3m}-\tsc{tr}-do/say-\tsc{rem.pst}=\tsc{3a.IO}\\
\ft He told them, ``this is your place, that cave is there for you, you go there whenever heavy rain comes,'' he said.
\exy \uex{``Oolagid nyoonoomb inin joogarra larda, arralboonjambaljin \newline mangir.''}
\gll ``Oola=gid nyoonoo=jamb i-ni-n joogarra larda, a-rr-alboo-n=jamb=al=jin mangir.''\\
water=\tsc{temp} there=\tsc{rel} \tsc{3m}-be.located-\tsc{cont} \tsc{2a.poss} bottom 1-\tsc{aug}-dig.out-\tsc{cont}=\tsc{rel}=\tsc{indef}=\tsc{3m.IO} always\\
\ft ``Your water is there down at the bottom, dig for it all the time.''
%\newpage
\exy \uex{Inganananirr,  ``barda arr ngandan gala Iwanyingan,'' injoonoojirr.}
\gll I-ng-ni-na-n=irr, ``barda arr nga-n-d-an gala Iwany-ngan,'' i-n-joo-na=jirr.\\
\tsc{3m}-\tsc{pst}-be.located-\tsc{rem.pst}-\tsc{cont}=\tsc{3a.DO} off go \tsc{1m}-\tsc{tr}-do/say-\tsc{cont} already Sunday.Island-\tsc{all} \tsc{3m}-\tsc{tr}-do/say-\tsc{rem.pst}=\tsc{3a.IO}\\
\ft He stayed with them, then he said, ``I'm going off to Sunday Island.''
\newpage\exy \uex{Danarnajirr  Iwanyoo Bilingbilinggo niimbal inamana.}
\gll Darr i-n-ar-na=jirr Iwany Bilinybiliny-go niimbal i-n-ma-na.\\
come \tsc{3m}-\tsc{tr}-pierce-\tsc{rem.pst}=\tsc{3a.IO} Sunday.Island Sunday.Island.law.ground-\tsc{abl} foot \tsc{3m}-\tsc{tr}-put-\tsc{rem.pst}\\
\ft He met the people on Sunday Island and walked from Bilinybiliny.
\exy \uex{Nyoononyi Goolinarr, inanggalanirr  bornkobornko boonyja.}
\gll Nyoonoo Goolinarr, i-na-ng-gala-n=irr bornkobornko boonyja.\\
other.side G. \tsc{3m}-\tsc{tr}-\tsc{pst}-visit-\tsc{cont}=\tsc{3a.DO} all.around everyone\\
\ft Then he was over on the other side (of Sunday Island), at Goolinarr, and he visited all the people there.
\exy \uex{Booloonggooloonggoo morr gir ingarralan,  ambooriny boonyja danarnajirr  inargarginirr,  injooloonganajirr  anggi gorna anggi loogal.}
\gll Booloonggooloonggoo morr gir i-ng-arr-ala-n, ambooriny boonyja darr i-n-ar-na=jirr i-n-argi-n=irr, i-n-jooloong-na=jirr anggi gorna anggi loogal.\\
B. path stay \tsc{3m}-\tsc{pst}-\tsc{aug}-visit-\tsc{cont} people all come \tsc{3m}-\tsc{tr}-pierce-\tsc{rem.pst}=\tsc{3a.IO} \tsc{3m}-\tsc{tr}-pick.up-\tsc{cont}=\tsc{3a.DO} \tsc{3m}-\tsc{tr}-collect-\tsc{rem.pst}=\tsc{3a.IO} what good what bad\\
\ft They were stopping on the way to Booloonggooloonggoo, he gathered all the people together, and told them what is good and what is bad.
\exy \uex{Ginyinggamba ``arralalan jiiba gorna, loogalba birarr arrin arra goo-lirrinyan.  Arra barn goolirrinyjin,  gorna jamb arral,'' injoonoojirr.}
\gll Ginyinggamba ``a-rr-alala-n jiiba gorna, loogal=b birarr a-rr-i-n arra goo-la-rr-inya-n. Arra barn goo-la-rr-joo-inyji-n, gorna jamb a-rr-al,'' i-n-joo-na=jirr.\\
that's.why 1-\tsc{aug}-follow-\tsc{cont} here good bad=\tsc{rel} leave.behind 1-\tsc{aug}-do/say-\tsc{cont} \tsc{neg} \tsc{2a}-\tsc{irr}-\tsc{aug}-catch-\tsc{cont} \tsc{neg} tell \tsc{2a}-\tsc{irr}-\tsc{aug}-do/say-\tsc{refl}-\tsc{cont} good \tsc{when} 1-\tsc{aug}-wander \tsc{3m}-\tsc{tr}-do/say-\tsc{rem.pst}=\tsc{3a.IO}\\
\ft Then he said, ``You look after the good things, and leave the bad things behind you. Don't tell yourselves what is good.''
\exy \uex{Nganyjal irrgoordoo lalin inanggalanan  Iwanyi, ginyinggo inamanajirr  ngaanka arar.}
\gll {Nganyjal irrgoordoo} lalin i-na-ng-gala-na-na Iwany, ginyinggo i-na-ma-na=jirr ngaanka arar.\\
we.don't.know.how.many summer \tsc{3m}-\tsc{tr}-\tsc{pst}-live-\tsc{cont}-\tsc{rem.pst} Sunday.Island from.there \tsc{3m}-\tsc{tr}-put-\tsc{rem.pst}=\tsc{3a.IO} language all.kinds\\
\ft We don't know how many years he stayed on Sunday Island, then he placed different languages on the Dampier Peninsula.
\exy \uex{``Barnanggarr lalin indan  booroo oondoordamba darra oonkarn  joo-garra goorlil,'' injoonoojirr.}
\gll ``Barnanggarr lalin i-n-joo-n booroo oondoord=jamb darr oo-n-k-ar-n joogarra goorlil,'' i-n-joo-na=jirr.\\
now summer \tsc{3m}-\tsc{tr}-do/say-\tsc{cont} time married.turtle=\tsc{rel} come \tsc{3m.irr}-\tsc{tr}-\tsc{fut.tr}-pierce-\tsc{cont} \tsc{2a.poss} turtle \tsc{3m}-\tsc{tr}-do/say-\tsc{rem.pst}=\tsc{3a.IO}\\
\ft ``Now the hot weather has come, married turtle will come for you,'' he said.
%\newpage
\exy \uex{``Lalin injoonoo  booroo jarramba joogarra lalin goorr aambanim joogarra boor arrala goorlil.}
\gll ``Lalin i-n-joo-na booroo jarri=jamb joogarra lalin goorr aamba-nim joogarra boor a-rr-jala-a goorlil.\\
summer \tsc{3m}-\tsc{tr}-do/say-\tsc{rem.pst} time this=\tsc{rel} \tsc{2a.poss} summer \tsc{2aug} man-\tsc{erg} \tsc{2a.poss} watch.out.for 2-\tsc{aug}-see-\tsc{fut} turtle\\
\ft ``You men should watch for turtle now that \mor{lalin} time has come to you.
\exy \uex{Aalga arra goolarrjarginjin  booroo oonggamarran garda arran jiiba garda arran jiiba goorlil jorro jamba nankarra jooboorl arriyangirr,''  injoonoojirr.}
\gll Aalga arra goo-la-rr-jargi-n jina booroo oo-ngg-marra-n gardi a-rr-ni-n jiiba gardi a-rr-ni-n jiiba goorlil jarri jamb nankarr joobool a-rr-joo-a-ng=irr,'' i-n-joo-na=jirr.\\
sun \tsc{neg} \tsc{2a}-\tsc{irr}-\tsc{aug}-be.afraid.of-\tsc{cont} \tsc{3m.IO} time \tsc{3m.irr}-\tsc{fut}-cook-\tsc{cont} still 2-\tsc{aug}-sit-\tsc{cont} here still 1-\tsc{aug}-sit-\tsc{cont} here turtle this \tsc{when} point.on.the.shore swim 1-\tsc{aug}-do/say-\tsc{fut}-appl.=\tsc{3a.DO} \tsc{3m}-\tsc{tr}-do/say-\tsc{rem.pst}=\tsc{3a.IO}\\
\ft ``Don't be frightened of the sun, it'll get hot, but do this and from the point jump into the water and chase after turtles,'' he told them.
\exy \uex{``Joorrinankoonjamb nankarra ginyingg booroo, ooranygid nyoonoo gir oonggarralan  garrin nyoonoomb Inboornoonan. Inboornoonan jarri gir oonggarrala  ambooriny,'' injoonoojirr.}
\gll ``Joorrinankoon=jamb nankarr ginyinggi booroo, oorany=gid nyoonoo gir oo-ngg-rr-ala-n garrin nyoonoo=jamb Inboornoonan Inboornoonan jarri gir oo-ngg-rr-gala-a ambooriny,'' i-n-joo-na=jirr.\\
J.=\tsc{rel} point.on.the.shore \tsc{3min} place woman=\tsc{temp} there live \tsc{3m.irr}-\tsc{fut}-\tsc{aug}-live-\tsc{cont} hill there=\tsc{rel} I. I. this stay \tsc{3m.irr}-\tsc{fut}-\tsc{aug}-live-\tsc{fut} people \tsc{3m}-\tsc{tr}-do/say-\tsc{rem.pst}=\tsc{3a.IO}\\
\ft ``This place is Joorrinankoon, the women will stop there on the hill at Inboornoonan. This place is Inboornoonan. There'll be people staying there,'' he told them.
\exy \uex{``Jarramb arralirr goorlil, biijib oonggoolooman joolganmolon \newline ingoorrngooloomana  garrin ginyingg morrgoon goorlil goowidany.''}
\gll ``jarri=jamb a-rr-jala=irr goorlil, biiji oo-ngg-loomoo-n joolganmolon i-ng-arr-ngoolooma-na garrin ginyinggi morr-goon goorlil goowidany.''\\
this=\tsc{rel} 2-\tsc{aug}-see=\tsc{3a.DO} turtle this.one.here \tsc{3m.irr}-\tsc{fut}-crawl-\tsc{cont} chasing.turtles \tsc{3m}-\tsc{pst}-\tsc{aug}-crawl-\tsc{rem.pst} hill \tsc{3min} path-\tsc{loc} turtle in.moonlight\\
\ft ``You'll see turtles. They'll crawl there when they've crawled up the hill, those married turtles will crawl there in the moonlight.''
\exy \uex{Jiindibirrgid goolboo injoolooloonan  moorrool inamanirr  garndijimbin garndijinbin inamoogarn booroo.}
\gll Jindibirr=gid goolboo i-n-jooloong-na-na moorrooloo i-na-ma-n=irr garndi-jimbin garndijinbin i-na-moogar-n booroo.\\
willy.wagtail=\tsc{temp} rock \tsc{3m}-\tsc{tr}-collect-\tsc{cont}-\tsc{rem.pst} small \tsc{3m}-\tsc{tr}-put-\tsc{cont}=\tsc{3a.DO} top-underneath one on top of the other \tsc{3m}-\tsc{tr}-make-\tsc{cont} place\\
\ft Willy wagtail put little rocks one on top of the other and built a nest.
\exy \uex{Irrinjamb  ginyinggi goolboo gard inamoogarna Jindibirrnim.}
\gll I-rr-ni-n=jamb ginyinggi goolboo i-na-moogar-na jindibirr-nim.\\
\tsc{3m}-\tsc{aug}-be.located-\tsc{cont}=\tsc{rel} \tsc{3min} rock \tsc{3m}-\tsc{tr}-make-\tsc{rem.pst} willy.wagtail-\tsc{erg}\\
\ft The rocks that willy wagtail built are still there.
\exy \uex{``Aamba barni morr arral oorany barni morr oonggarral,''  injoonoojirr.}
\gll ``Aamba barni morr a-rr-al oorany barni morr oo-ngg-a-rr-al,'' i-n-joo-na=jirr.\\
man this way 2-\tsc{aug}-wander woman this.(way) path \tsc{3m.irr}-\tsc{fut}-\tsc{tr}-\tsc{aug}-wander \tsc{3m}-\tsc{tr}-do/say-\tsc{rem.pst}=\tsc{3a.IO}\\
\ft ``Men, you go this way, and women, you go that way,'' he told them.
\exy \uex{``Aamba jirr booroo, arra ooranynim arra goolarralan,''  injoonoojirr.}
\gll ``aamba jirr booroo, arra oorany-nim arra goo-la-rr-ala-n,'' i-n-joo-na=jirr.\\
man \tsc{3a.poss} place \tsc{neg} woman-\tsc{erg} \tsc{neg} \tsc{2a}-\tsc{irr}-\tsc{aug}-visit-\tsc{cont} \tsc{3m}-\tsc{tr}-do/say-\tsc{rem.pst}=\tsc{3a.IO}\\
\ft ``Women, you don't go to men's places,'' he told them.
\nt DW didn't have the first \mor{arra} in the original text.
\exy \uex{Moorrooloo Boonggoolalmoorroon inamana  riiwa. Oola boowaybooway inman nyalab.}
\gll {Moorrooloo Boonggoolalmarroon} i-n-ma-na riiwa. oola boowaybooway i-n-ma-n nyalab.\\
Beach.on.Sunday.Island \tsc{3m}-\tsc{tr}-put-\tsc{rem.pst} hole water bubble.up \tsc{3m}-\tsc{tr}-put-\tsc{cont} over.there\\
\ft At Moorrooloo Boonggoolalmoorroon he put a hole. Water bubbles up there.
\exy \uex{``Arra jooroorr goolirrin  jiibadan booroo bardagang. Baawa orrolboonirr,  jooroorr arrin ginyinggi booroo riiwa boordiji oola darr oonkarn  joogarra, bardoonoo oonggilgin  boordiji,'' injoonoojirr.}
\gll ``arra jooroorr goo-li-rr-i-n jiiba booroo bardag-ng. baawa o-rr-olboo-n=irr, jooroorr a-rr-i-n ginyinggi booroo riiwa boordij oola darr oo-n-k-ar-n joogarra, bardoon oo-ngg-ilgi-n boordij,'' i-n-joo-na=jirr.\\
\tsc{neg} poke \tsc{2a}-\tsc{irr}-\tsc{aug}-do/say-\tsc{cont} here place tree-\tsc{ins} child 1-\tsc{aug}-dig.out-\tsc{cont}=\tsc{3a.DO} poke 1-\tsc{aug}-do/say-\tsc{cont} \tsc{3min} place hole big water come \tsc{3m.irr}-\tsc{tr}-\tsc{fut.tr}-pierce-\tsc{cont} \tsc{2a.poss} southeast.wind \tsc{3m.irr}-\tsc{fut}-blow-\tsc{cont} big \tsc{3m}-\tsc{tr}-do/say-\tsc{rem.pst}=\tsc{3a.IO}\\
\ft ``Don't poke this hole. If you kids play with sticks and poke this place, a big rainstorm will come to you, and the wind will blow strong,'' he told them.
%\nt riiwa boordiji is he phrasing on tape.
%\newpage
\exy \uex{Ginyinggo ``wiira ngankal jarri birarrjamb ngankiyanggoorr,''  injoonoojirr.}
\gll Ginyinggo ``wiira nga-n-gala jarri birarr=jamb nga-n-k-i-a-ng=goorr,'' i-n-joo-na=jirr.\\
from.there little.while \tsc{1m}-\tsc{tr}-live this leave.behind=\tsc{rel} \tsc{1m}-\tsc{tr}-\tsc{fut.tr}-\tsc{2m}-\tsc{fut}-\tsc{ins}=\tsc{2a.DO} \tsc{3m}-\tsc{tr}-do/say-\tsc{rem.pst}=\tsc{3a.IO}\\
\ft Then he said, ``I'm going to leave you in a little while.''
\newpage\exy \uex{``Nyirra boorajii?'' injoonoojirr.}
\gll ``Nyirra boora=jiy?'' i-n-joo-na=jirr.\\
which place=\tsc{2m.poss} \tsc{3m}-\tsc{tr}-do/say-\tsc{rem.pst}=\tsc{3a.IO}\\
\ft ``Where are you going?'' They asked him,
\exy \uex{``Arra! bard arr ngandan garndingan gooloo ngankalanjarran,  \newline Nyiinya.}
\gll ``Arra! barda arr nga-n-d-an garndi-ngan gooloo nga-n-kala-n=jarran, Nyiinya.\\
hey! away go \tsc{1m}-\tsc{tr}-do/say-\tsc{cont} top-\tsc{all} father \tsc{1m}-\tsc{tr}-visit-\tsc{cont}=\tsc{1m.IO} Ny.\\
\ft ``I'm going to heaven; I'll go and see our father Nyiinya.''
\exy \uex{Jiib inin garndi. Ginyingginim oonkalalankoorr  gorndo agal \newline ngayoonim ngankalalankoorr  gorndo,'' injoonoojirr.}
\gll Jiiba i-ni-n garndi. ginyinggi-nim oo-n-k-galala-n=goorr gorndi-o agal ngayoo-nim nga-n-k-galala-n=goorr gornd-o,'' i-n-joo-na=jirr.\\
here \tsc{3m}-sit-\tsc{cont} top \tsc{3min}-\tsc{erg} \tsc{3m.irr}-\tsc{tr}-\tsc{fut.tr}-follow-\tsc{cont}=\tsc{2a.DO} top-\tsc{abl} and \tsc{1min}-\tsc{erg} \tsc{1m}-\tsc{tr}-\tsc{fut.tr}-follow-\tsc{cont}=\tsc{2a.DO} top-\tsc{abl} \tsc{3m}-\tsc{tr}-do/say-\tsc{rem.pst}=\tsc{3a.IO}\\
\ft ``He lives in heaven. He'll look after you from the sky and I'll look after you from the sky,'' he said to them.
%\newpage
\exy \uex{Ingarrmalinyjin  boonyja.}
\gll I-ng-arr-m-al-inyji-n boonyja.\\
\tsc{3m}-\tsc{pst}-\tsc{aug}-\tsc{refl}-look.at-\tsc{refl}-\tsc{cont} everyone\\
\ft They all looked at each other.
\exy \uex{``Nyirroogoordoo morr oonggiidi barda angginyarr wirr oonggarrm?'' ingirrinijin.}
\gll ``Nyirroogoordoo morr oo-ngg-iidi barda anggi-nyarr wirr oo-ngg-arrm?'' i-ng-irr-i-ni=jin.\\
how path \tsc{3m.irr}-\tsc{fut}-go away what-\tsc{comit} go.up \tsc{3m.irr}-\tsc{fut}-get.up \tsc{3m}-\tsc{pst}-\tsc{aug}-do/say-\tsc{rem.pst}=\tsc{3m.IO}\\
\ft ``How will he go? What will he use to rise up?'' they asked themselves.
\exy \uex{Gard ingarrbooloonanajin.}
\gll Gardi i-ng-arr-booloo-na-na=jin.\\
still \tsc{3m}-\tsc{pst}-\tsc{1aug}-disbelieve-\tsc{cont}-\tsc{rem.pst}=\tsc{3m.IO}\\
\ft They still disbelieved him.
\exy \uex{Gooyarra goowidi inganana,  gala barnkarda ``arrangan jarr amboonoo nganjoonoongjarrgoorr,  arralalan anggirrgoord nganamana joogarra gorna, loogal gala goorrmoonggoon,'' injoonoojirr.}
\gll Gooyarra goowidi i-ng-ni-na, gala ``a-rr-nga-n jarri amboon nga-n-joo-noo-ng=jarrgoorr, a-rr-alala-n anggirrgoord nga-na-ma-na joogarra gorna, loogal gala goorr-moonggoon,'' i-n-joo-na=jirr.\\
two month \tsc{3m}-\tsc{pst}-be.located-\tsc{rem.pst} well.then 1-\tsc{aug}-be-\tsc{cont} this together \tsc{1m}-\tsc{tr}-do/say-\tsc{rem.pst}-\tsc{ins}=\tsc{2a.DO} 1-\tsc{aug}-follow-\tsc{cont} whatever \tsc{1m}-\tsc{tr}-put-\tsc{pst} \tsc{2a.poss} good bad well.then \tsc{2a}-know \tsc{3m}-\tsc{tr}-do/say-\tsc{rem.pst}=\tsc{3a.IO}\\
\ft Eventually, two months later, (he said) ``We've been together for a long time. I've stayed with you and looked after you, I've given you all sorts of things, and you know what's good and what's bad,'' he said to them.
\exy \uex{Ginyinggo ingarralalan  ``wayi!'' injoonoojirr.}
\gll Ginyinggo i-ng-arr-galala-na ``way!'' i-n-joo-na=jirr.\\
from.there \tsc{3m}-\tsc{pst}-\tsc{aug}-follow-\tsc{rem.pst} Come.on! \tsc{3m}-\tsc{tr}-do/say-\tsc{rem.pst}=\tsc{3a.IO}\\
\ft Then they followed him. ``Come here!'' he said to them.
\exy \uex{Roowil ingirrinyanang  barda Iilonko barda Jawanan daab ingirrinyan.}
\gll Roowil i-ng-arr-inya-na-ng barda Iilon-go barda Jawanan daab i-ng-arr-inya-n.\\
walk \tsc{3m}-\tsc{pst}-\tsc{aug}-catch-\tsc{rem.pst}-\tsc{appl} off Ii.-\tsc{abl} off J. climb.up \tsc{3m}-\tsc{pst}-\tsc{aug}-catch-\tsc{cont}\\
\ft They walked from Iilon to Jawanan and they climbed up.
\newpage\exy \uex{``Nyirrabooroo inkajanmoord?''  ingirrminyjin  ambooriny.}
\gll ``nyirra-booroo i-n-kaja-n=moordoo?'' i-ng-arr-m-i-nyji-n ambooriny.\\
which-place \tsc{3m}-\tsc{tr}-take-\tsc{cont}=\tsc{1aug.DO} \tsc{3m}-\tsc{pst}-\tsc{aug}-\tsc{refl}-do/say-\tsc{refl}-\tsc{cont} people\\
\ft ``Where is he taking us?'' the people thought.
\exy \uex{Roowil ingirrinyana  Ngalngoorarra yoorr ingarraman.  Nyoonoo \newline Balalbalalngarr daab ingirrinyan  goona goorriron.}
\gll Roowil i-ng-arr-inya-na Ngalngoorarra yoorr i-ng-arr-ma-na. nyoonoo Balalbalalngarr daab i-ng-arr-inya-n goona goorrir-goon.\\
walk \tsc{3m}-\tsc{pst}-\tsc{1aug}-catch-\tsc{rem.pst} Ng. come.down \tsc{3m}-\tsc{pst}-\tsc{1aug}-put-\tsc{rem.pst} there B. climb.up \tsc{3m}-\tsc{pst}-\tsc{aug}-catch-\tsc{cont} further.on fig.tree-\tsc{loc}\\
\ft They walked to Ngalngoorarra and came down at Balalbalalngarr, and climbed up at the big fig tree.
\exy \uex{Goodil ingarranan  barda Boolgoonngan. Daab innyan  boordiji niya Boolgoon.}
\gll Goodil i-ng-arr-ga-na barda Boolgoon-ngan. daab i-n-nya-na boordij niya Boolgoon.\\
turn \tsc{3m}-\tsc{pst}-\tsc{aug}-take-\tsc{rem.pst} off B.-\tsc{all} climb \tsc{3m}-\tsc{tr}-catch-\tsc{rem.pst} big ridge.of.hill B.\\
\ft They turned off for Boolgoon and climbed to Boolgoon, which is a big hill.
%\newpage
\exy \uex{Ingarralalan,  ingarrmalinyjin  ambooriny jirrjirr injoonoo  garndi.}
\gll I-ng-arr-alala-na, i-ng-arr-m-al-inyji-n ambooriny jirrjirr i-n-joo-na garndi.\\
\tsc{3m}-\tsc{pst}-\tsc{aug}-watch-\tsc{rem.pst} \tsc{3m}-\tsc{pst}-\tsc{aug}-\tsc{refl}-see-\tsc{refl}-\tsc{cont} people stand \tsc{3m}-\tsc{tr}-do/say-\tsc{rem.pst} top\\
\ft The people looked at him, they looked at each other as he stood on top of the hill.
\newpage\exy \uex{Jiin inarna  booroo bardagang, ``Arra anganagij goowidi,'' injoonoojirr.}
\gll Jiin i-n-ar-na booroo bardag-ng, ``Arra angan=gij goowidi,'' i-n-joo-na=jirr \\
point.at \tsc{3m}-\tsc{tr}-pierce-\tsc{rem.pst} place tree-\tsc{ins} \tsc{neg} near=very moon \tsc{3m}-\tsc{tr}-do/say-\tsc{rem.pst}=\tsc{3a.IO}\\
\ft He pointed at the place with a stick. ``No, the moon is too close,'' he said to them.
\exy \uex{``Barda jarda anggoorroondoorroo iinalangngan barda.'' Garrjadiny iinalangngan ginyinggon ingoorroondoorroon  boonyjanim ingarralalana  daab ingirrinyan  Garrjadiny.}
\gll ``Barda jarda a-ngg-rr-goondoorroo iinalang-ngan barda.'' Garrjadiny iinalang-ngan ginyinggon i-ng-arr-goondoorra-n boonyja-nim i-ng-arr-galala-na daab i-ng-arr-inya-n Garrjadiny.\\
off \tsc{1a.poss} 1-\tsc{fut}-\tsc{aug}-cross.over island-\tsc{all} off G. island-\tsc{all} then \tsc{3m}-\tsc{pst}-\tsc{aug}-cross.over-\tsc{cont} everyone-\tsc{erg} \tsc{3m}-\tsc{pst}-\tsc{aug}-follow-\tsc{rem.pst} go.ashore \tsc{3m}-\tsc{pst}-\tsc{aug}-catch-\tsc{cont} G.\\
\ft ``We'll cross over to the island.'' They all followed him and crossed over to the Garrjadiny islands and climbed Garrjadiny.
\exy \uex{Jirrjirr injoonoo  niyamarr garndi goolboon, boordijon goolb.}
\gll Jirrjirr i-n-joo-na niya goolboo-goon boordij-goon goolboo.\\
stand.up \tsc{3m}-\tsc{tr}-do/say-\tsc{rem.pst} right.on.top rock-\tsc{loc} big-\tsc{loc} rock\\
\ft He stood right on top, on a big rock.
\exy \uex{``Nganjoomarrabjoogarra  galamb birarr ngandankoorr.  Arra goolarr-ganyjin  anggirrgoordoo loogalbal birarr arrin aarlimaybijoogarra oola joogarra,'' injoonoojirr.}
\gll ``Nga-n-joo-marr=b=joogarra gala=jamb birarr nga-n-da-an=koorr. Arra goo-la-rr-ganyji-n anggirrgoordoo loogal=b=al birarr arr-joo-n aarlimay=b=joogarra oola joogarra,'' i-n-joo-na=jirr.\\
\tsc{1m}-\tsc{tr}-do/say-\tsc{sembl}=\tsc{rel}=\tsc{2a.IO} already=\tsc{rel} leave.behind \tsc{1m}-\tsc{tr}-do/say-\tsc{cont}=\tsc{2a.DO} \tsc{neg} \tsc{2a}-\tsc{irr}-\tsc{aug}-forget-\tsc{cont} whatever bad=\tsc{rel}=\tsc{indef} leave.behind \tsc{1aug}-do/say-\tsc{cont} food=\tsc{rel}=\tsc{2a.poss} water \tsc{2a.poss} \tsc{3m}-\tsc{tr}-do/say-\tsc{rem.pst}=\tsc{3a.IO}\\
\ft ``I told you the time that I would leave you. You shouldn't forget anything. Leave bad food and water behind,'' he told them
\exy \uex{``Oomban joogarra oola. Biidini joogarrirr oola. Niiminiimba joogarra oola joogarra oola arang yaaga joodinygo oonkan,''  injoonoojirr.}%\enlargethispage*{1cm}
\gll ``Oomban joogarra oola. Biidin joogarra=irr oola. Niiminiimi=b joogarra oola joogarra oola arang yaaga joodiny oo-n-k-i-n,'' i-n-joo-na=jirr.\\
fresh.water.soak \tsc{2a.poss} water fresh.water \tsc{2a.poss}=\tsc{3a} water sand.soakage=\tsc{rel} \tsc{2a.poss} water \tsc{2a.poss} water other hole always \tsc{3m}-\tsc{tr}-\tsc{fut}-be-\tsc{cont} \tsc{3m}-\tsc{tr}-say-\tsc{rem.pst}=\tsc{3a.IO}\\
\ft ``You've got soakages for your water. You've got soakages in the sand and other water, pools---they'll always be there,'' he said to them.
\exy \uex{``Arralalan booroo gorna, ngananamarrabagoorr  booroo,'' injoonoojirr.}
\gll ``A-arr-jalala-n booroo gorna, nga-na-na-marr=ba=goorr booroo,'' i-n-joo-na=jirr.\\
1-\tsc{1aug}-look.after-\tsc{cont} place good \tsc{1m}-\tsc{tr}-\tsc{tr}-\tsc{sembl}=\tsc{rel}=\tsc{2a.DO} place \tsc{3m}-\tsc{tr}-do/say-\tsc{rem.pst}=\tsc{3a.IO}\\
\ft ``Look after  the places I gave you well,'' he told them.
\exy \uex{``Aarli joogarra ginyingg, arar aarli loogal, arar aarli joodinygo arrarlin layiidi joodinygo.}
\gll ``Aarli joogarra ginyinggi, arar aarli loogal, arar aarli joodiny a-rr-rli-n laya joodiny.\\
fish \tsc{2a.poss} \tsc{3min} all.kinds fish bad all.kinds fish for.ever 1-\tsc{aug}-eat-\tsc{cont} fatty for.good\\
\ft ``This is your fish, some fish are bad, some fish you can always eat: they're always fat.
\exy \uex{Ara loogal oonkin  biindany oonkin.  Ara layiid oonkin.}
\gll Ara loogal oo-n-k-i-n biindany oo-n-k-i-n. Ara laya oo-n-k-i-n.\\
other old \tsc{3m.irr}-\tsc{tr}-\tsc{fut.tr}-do/say-\tsc{cont} not.fatty \tsc{3m.irr}-\tsc{tr}-\tsc{fut.tr}-do/say-\tsc{cont} other fatty \tsc{3m.irr}-\tsc{tr}-\tsc{fut.tr}-do/say-\tsc{cont}\\
\ft Others are bad and don't have good fat. Others will get fat.
\exy \uex{Ginyinggon lalin ginyinggon barrgana ginyinggon jalalay boonyjamb arar joogarr aarli,'' injoonoojirr.}
\gll Ginyinggon lalin ginyinggon barrgan ginyinggon jalalay boonyja=jamb arar joogarra aarli,'' i-n-joo-na=jirr.\\
then summer then cold.season then warm.up.time all=\tsc{rel} all.kinds \tsc{2a.poss} fish \tsc{3m}-\tsc{tr}-do/say-\tsc{rem.pst}=\tsc{3a.IO}\\
\ft Then it will be \mor{lalin} season (married turtle season, in November) and the South-East wind will blow, then \mor{jalalayi} season is all different,'' he told them.
\exy \uex{``Galamb arr ngandan biila biilamb darr ngankarajoogarra.  Laamboo ngankimijoogarra,''  injoonoojirr.}
\gll ``Gal=jamb arr nga-n-d-an biila biila=jamb darr ng-an-k-ar-a=joogarra. Laamboo ng-an-k-i-mi=joogarra,'' i-n-joo-na=jirr.\\
wander=\tsc{rel} go \tsc{1m}-\tsc{tr}-do/say-\tsc{cont} again also=\tsc{rel} come \tsc{pst}-\tsc{tr}-\tsc{fut.tr}-pierce-\tsc{fut}=\tsc{2a.IO} later \tsc{pst}-\tsc{tr}-\tsc{fut.tr}-\tsc{3m}-search.for=\tsc{2a.IO} \tsc{3m}-\tsc{tr}-do/say-\tsc{rem.pst}=\tsc{3a.IO}\\
\ft ``Right, I'm going but I will come back again to you. I'll come back later,'' he told them.
\exy \uex{Ginyinggo boor ingarralan  jarr ingarralan  oola ingoomoogarinyjina  jirrjirrb injoonoo.}
\gll Ginyinggo boor i-ng-arr-jala-n jarri i-ng-arr-jala-n oola i-ng-moogar-inyji-na jirrjirr=b i-n-joo-na.\\
from.there look.around \tsc{3m}-\tsc{pst}-\tsc{1aug}-look.at.self-\tsc{cont}  this \tsc{3m}-\tsc{pst}-\tsc{1aug}-look.at-\tsc{cont} cloud \tsc{3m}-\tsc{pst}-make-\tsc{refl}-\tsc{rem.pst} stand=\tsc{rel} \tsc{3m}-\tsc{tr}-do/say-\tsc{rem.pst}\\
\ft Then they saw that clouds formed (themselves) where he was standing.
\exy \uex{Wirr inyjarrmin ``galamb arr ngandan. Arra goolarrganyj anggirr-goord nganama joogarra ngananagoorr,''  injoonoojirr.}
\gll Wirr i-ny-jarrmi-n ``gala=jamb arr nga-n-d-an. Arra goo-la-rr-ganyji anggirrgoord nga-na-ma joogarra ng-arnan-a-goorr,'' i-n-joo-na=jirr.\\
go.up \tsc{3m}-\tsc{pst}-rise-\tsc{cont} well.then=\tsc{rel} go \tsc{1m}-\tsc{tr}-do/say-\tsc{cont} \tsc{neg} \tsc{2a}-\tsc{irr}-\tsc{aug}-forget whatever \tsc{1m}-\tsc{tr}-put \tsc{2a.poss} 1-\tsc{tr-}[give-\tsc{rem.pst=2a.DO} \tsc{3m}-\tsc{tr}-do/say-\tsc{rem.pst}=\tsc{3a.IO}\\
\ft He rose up, ``I'm going now. Don't forget anything I put here and gave you,'' he said to them.
\exy \uex{``Galab arr ngandan, joorrgo,'' injoonoojirr.}
\gll ``Gala=b arr nga-n-d-an, joorrgo,'' i-n-joo-noo=jirr.\\
well.then=\tsc{rel} go \tsc{1m}-\tsc{tr}-do/say-\tsc{cont} goodbye! \tsc{3m}-\tsc{tr}-do/say-\tsc{rem.pst}=\tsc{3a.IO}\\
\ft ``I'm going. Goodbye,'' he said.
\exy \uex{Wirr inyjarrmin oolon barda garndingan.}
\gll Wirr i-ny-jarrmi-n ool-on barda garndi-ngan.\\
go.up \tsc{3m}-\tsc{pst}-rise-\tsc{cont} cloud-\tsc{loc} off above-\tsc{all}\\
\ft He rose up on a cloud off to heaven.
\end{exye}

\subsection{BDI1: How we know that Bardi people have always been here}
This narrative was given by Jessie Sampi in 2001, in response to my questions about the Bardi origin legends reported in \citet{dix96}. The sources of those legents are not given in that article and we have been unable to trace them; this is discussed further in \sref{ormyth} above. This text sets out the reasons why Bardi people should be considered the long-standing traditional owners of Bardi country.

\newpage\setcounter{exxy}{0}\begin{exye}
\exy \uex{Milimil ingorron  amboorinynim araboorarda darr angarrarana,  Bardi ambooriny ininggijing  milimilon. Arramba.}
\gll Milimil i-ng-orr-o-n ambooriny-nim ara booroo=barda darr a-ng-arr-ar-ana, Bardi ambooriny i-ni-ng=gij-ng milimil-on. Arra=amb.\\
write \tsc{3m}-\tsc{pst}-\tsc{aug}-poke-\tsc{cont} people-\tsc{erg} another place=maybe come 1-\tsc{pst}-\tsc{aug}-pierce-\tsc{rem.pst} Bardi people \tsc{3m}-be.located-inst=very-inst paper-\tsc{loc} \tsc{neg}=\tsc{rel}\\
\ft People have written that Bardi people came from somewhere else, a long time ago. But it's not true.
\exy \uex{Arra ningarrard ngaliyijin ngaynim.}
\gll Arra ningarrarda nga-li-yi=jin ngay-nim.\\
\tsc{neg} believe \tsc{1m}-\tsc{irr}-do/say=\tsc{3m.IO} \tsc{1min}-\tsc{erg}\\
\ft I don't believe that.
\exy \uex{Bardi ambooriny, aarlimay, aarlibarnangg, oola, gardamb ingarrala-najirr  irr ambooriny aarl agal goorlil, nimoonggoon anjoon  goorrir anggirrgoordoo, ginyinggon joonamb ingarrarlan.}
\gll Bardi ambooriny, aarlimay, aarlibarnangg, oola, gard=amb i-ng-arr-ala-na=jirr irr ambooriny aarl agal goorlil, ni-moonggoon a-n-joo-n goorrir anggirrgoordoo, ginyinggon nyoon=amb i-ng-arr-a-rl-an.\\
Bardi people food shellfish water already=\tsc{rel} \tsc{3m}-\tsc{pst}-\tsc{1a-}see-\tsc{rem.pst}=\tsc{3a.IO} they people fish and turtle learn 1-\tsc{tr}-do/say-\tsc{cont} fig.tree whatever then there=\tsc{thus} 3-\tsc{pst-aug-tr}-eat-\tsc{cont}\\
\ft Bardi people know about bush food, fish, the water and tides, they've been living with this food, they know about these things and we are used to eating all these things.
\exy \uex{Arroodoomoordoo ginyingg arroodoo marrirborla, arrmoonggoon \linebreak jawal, arrmoonggoon jamba gamardajamoo jardarr ingarrananamoordoo,  aarlibarnangga, aarlimay, arrmoonggoon ingarramana,  anggingan marlin, arra  gayaryoon wiliwil angoorrooloonganan,  gardilinya aarli, anggirrgoorda, gardo angirrinyanan  arrmoonggoonb ingarramanagij  jamoogamardanimjardirr.}
\gll Arroodoo=moordoo ginyinggi arroodoo marrir-borla, arr-moonggoon jawal, arr-moonggoon jamba gamarda-jamoo jarda=rr i-ng-arra-na-na=moord, aarlibarnangg, aarlimay, arr-moonggoon i-ng-arr-a-ma-na, anggi-ngan ma-rli-n, arra  gayar-yoon wiliwili a-ng-oorr-ooloong-na-n, gardiliny aarli, anggirrgoorda, gardo a-ng-arr-inya-na-n arr-moonggoon=b i-ng-arr-a-ma-na=gij jamoo gamarda-nim=jard-irr.\\
\tsc{1aug}=\tsc{1aug.DO} \tsc{3min} \tsc{1aug} older.sister-younger.sister \tsc{1aug}-know story \tsc{aug}-know \tsc{when} maternal.grandmother-maternal.grandfather \tsc{1aug.poss}=\tsc{3a} \tsc{3m}-\tsc{pst}-\tsc{aug}-give-\tsc{cont}-\tsc{rem.pst}=\tsc{1a.DO} fish food \tsc{aug}-know \tsc{3m}-\tsc{pst}-\tsc{aug}-put-\tsc{rem.pst} what-\tsc{all} \tsc{ger}-eat-\tsc{cont} \tsc{neg}
white.person-\tsc{source} fishing.line 1-\tsc{pst}-\tsc{aug}-collect-\tsc{cont}-\tsc{rem.pst} monkey.fish fish whatever still 1-\tsc{pst}-\tsc{aug}-get-\tsc{cont}-\tsc{rem.pst} \tsc{1a}-know=\tsc{rel} \tsc{3m}-\tsc{pst}-\tsc{aug}-\tsc{tr}-put-\tsc{rem.pst}=very maternal.grandfather maternal.grandmother-\tsc{erg}=\tsc{1a.IO}=\tsc{3a}\\
\ft Us people, we know the stories our grandparents taught us, about shellfish and fish: they taught us what to eat. We didn't go fishing with white people's fishing lines; we caught monkeyfish and all sorts of things, and we still know those things which our grandparents taught us.
\exy \uex{Arramb ningarrard alarramijirr  gayarjoon, ``ara boor darr goongarrarayin  nyalab,'' ininba  milimilgoondarr.}
\gll Arra=jamb ningarrarda a-la-rr-a-ma=jirr gayar-joon, ``ara booroo darr goo-ng-arr-a-ra=jin nyalab,'' i-ni-n=ba milimili-goondarr.\\
\tsc{neg}=\tsc{rel} believe 1-\tsc{irr}-\tsc{aug}-put=\tsc{3a.IO} white.person-\tsc{source} another place come \tsc{2a}-\tsc{pst}-\tsc{aug}-\tsc{tr}-pierce=\tsc{3m.IO} over.there \tsc{3m}-be.located-\tsc{pres}=\tsc{rel} book-\tsc{loc_2}\\
\ft That's why we don't believe those white people when they say, ``you came here from somewhere else'' and when they write it on paper.
\exy \uex{Gardamoord jarri Iwanyoon jardirr jamoogamard  arroodoob jardirr, (anggi) bardoonoo garadard angarralinang  gayar, nyalab arr injoona  aamba ginyinggarda jarda jamoo, angg inamanajard gamard jarramba malarrngan.}
\gll Garda=moord jarri Iwany-oon jard-irr jamoo-gamarda arroodoo=b jarda, (anggi) bardoon garadard a-ng-arr-ali-na-ng gayar, nyalab arr i-n-joo-na aamba ginyingg=arda jarda jamoo, angg i-n-ma-na=jard gamarda jarri=jamb malarr-ngan.\\
still=\tsc{1a.DO} this Sunday.Island-\tsc{loc} \tsc{1a.poss-3a} maternal.grandfather-maternal.grandmother \tsc{1aug}=\tsc{rel} \tsc{1aug.poss} what skin lightskinned 1-\tsc{pst}-\tsc{aug}-\tsc{tr}-move-\tsc{rem.pst}-inst white.person over.there go \tsc{3m}-\tsc{tr}-do/say-\tsc{rem.pst} man \tsc{3min}=\tsc{interrog} \tsc{1a.poss} maternal.grandfather something \tsc{3m}-\tsc{tr}-put-\tsc{rem.pst}=\tsc{1a.IO} maternal.grandmother this=\tsc{rel} wife-\tsc{all}\\
\ft Our grandparents were on Sunday Island and a light-skinned man came to visit them, and he made our grandmother his girlfriend.
\exy \uex{Ginyinggamba jard birrii boolgar, bard gooloo jarri jard maanka.}
\gll Ginyinggamba jarda birrii boolgar, barda gooloo jarri jarda maanka.\\
and that's why \tsc{1a.poss} mother white but father this \tsc{1aug.poss} black\\
\ft That's why our mother's white, but our father's black.
%\newpage
\exy \uex{Gard Iwanyoon, jarr anggoobooryoon, Barnaradooyoon ngoonyjoon, gala goorrmoonggoon Borrgoron jiiba agal Marnbij Ardinoogoon \newline ginyinggiyoonamb jarda gooloo arroodoo.}
\gll Gardi Iwany-oon, jarri anggooboor-yoon, Barnaradooyoon ngoonyjoon, gala goorr-moonggoon Borrgoron jiiba agal Marnbij Ardinoogoon ginyinggi-yoon=jamb jarda gooloo arroodoo.\\
already Sunday.Island-\tsc{loc} this somewhere-\tsc{source} B.-\tsc{source} only.this now 2\tsc{a}-knowledge Brown's.homestead here and Mission.Bay Shenton.Bluff \tsc{3min}-\tsc{source}=\tsc{rel} \tsc{1a.poss} father \tsc{1aug}\\
\ft Our father was from Barnarad area, Borrogoron, Marnbij, and Ardinoogoon area.
\exy \uex{Barnanggarrgidi, arramba irrmoonggoon baybirr arrab jardirr jamoo-gamarda baawa jardirr.}
\gll Barnanggarr=gid, arra=jamb irr-moonggoon baybirr arra=b jard=irr jamoo-gamarda baawa jarda.\\
now=\tsc{temp} \tsc{neg}=\tsc{rel} \tsc{3a-}know behind \tsc{neg}=\tsc{rel} \tsc{1a.poss}=\tsc{3a} maternal.grandfather-maternal.grandmother child 1+2\tsc{a}\\
\ft Nowadays our children don't know the things  our grandparents taught us.
\exy \uex{Gala gayarmarr irral wiliwil irroongooloong agal anggirrgoord \newline gayarmarr.}
\gll Gala gayar-marr i-rr-al wiliwili i-arr-a-ngooloo-ng agal anggirrgoord gayar-marr.\\
now white.person-\tsc{sembl} 3-\tsc{aug}-live fishing.line \tsc{3m}-\tsc{aug}-\tsc{tr}-throw-\tsc{appl} and all.that white.person-\tsc{sembl}\\
\ft They fish and throw their lines and do other things like white people.
\exy \uex{Galamb niindoo barnkardardamb.}
\gll Gal=amb niindoo barnkard=ard=jamb.\\
well=\tsc{rel} maybe end=\tsc{interrog}=\tsc{rel}\\
\ft I think it's all finished now.
\exy \uex{Arrmoonggoon nyirroogoordoo ingarralalanjardirr  ambooriny milon.}
\gll Arr-moonggoon nyirragoordoo i-ng-arr-alal-an=jard=irr ambooriny milon.\\
\tsc{1a}-know how \tsc{3m}-\tsc{pst}-\tsc{aug}-follow-\tsc{cont}=\tsc{1a.IO}=\tsc{3a.DO} people long.ago\\
\ft We are the only ones left who know how they used to live.
\newpage\exy \uex{Gala barnkarda.}
\gll Gala barnkarda.\\
that's.it\\
\ft That's the end.
\end{exye}

\subsection{LPB: Story about a crocodile at Pender Bay}
Jessie Sampi told this story to Claire Bowern in 2001 after a fishing trip to Pender Bay, on the western side of the Dampier Peninsula.\index{Sampi, Jessie}\index{Pender Bay}

\setcounter{exxy}{0}\begin{exye}
\exy \uex{Booroo nyanbirroonony.}
\gll Booroo nyanbirroonony.\\
place other.side\\
\ft It's a place on the other side.
\exy \uex{Bardi bard arr angirrin  Goorrbarlgoonngan.}
\gll Bardi barda arr a-ng-irr-i-n Goorrbarlgoon-ngan.\\
Bardi away go 1-\tsc{pst}-\tsc{aug}-do/say-\tsc{cont} {Pender.Bay}-\tsc{all}\\
\ft Yesterday we went to Pender Bay.
\exy \uex{Nyoonamba jiiny nganarjin booroo.}
\gll Nyoonoo=amb jiiny ng-n-ar=jina booroo.\\
there=\tsc{rel} point.at \tsc{pst}-\tsc{tr}-pierce=\tsc{1m.poss} place\\
\ft I pointed out a place.
\exy \uex{Nganjilngijirr jawal amboorinynim ingirrilngananajard.}
\gll Nga-n-jilngi=jirr jawal ambooriny-nim i-ng-irr-ilngi-na-na=jard.\\
\tsc{1m}-\tsc{tr}-tell=\tsc{3a.IO} story people-\tsc{erg} \tsc{3m}-\tsc{pst}-\tsc{aug}-tell-\tsc{cont}-\tsc{rem.pst}=\tsc{1a.IO}\\
\ft I told a story that people used to tell us.
\exy \uex{Booroo garndi boordijangarr oola inin.}
\gll Booroo garndi boordij oola i-ni-n.\\
place top really.big water \tsc{3m}-be.located-\tsc{cont}\\
\ft Over there there's a big lagoon.
\newpage\exy \uex{Oolab inarn  wiinya indanjamb.}
\gll Oola=b i-n-ar-n wiinya i-n-d-an=jamb.\\
rain=\tsc{rel} \tsc{3m}-\tsc{tr}-pierce-\tsc{cont} full \tsc{3m}-\tsc{tr}-do/say-\tsc{cont}=\tsc{rel}\\
\ft When it rains it becomes full.
\exy \uex{Linygoorroogid nyalab arr indan  daab indan  nyoonoo o, ininjamba  ginyingg arinyjangarr linygoorr.}
\gll Linygoorroo=gid nyalab arr i-n-d-an daab i-n-d-an nyoonoo o, i-ni-n=jamb ginyinggi arinyj=angarr linygoorr.\\
crocodile=\tsc{temp} from.this.side go \tsc{3m}-\tsc{tr}-do/say-\tsc{cont} climb.up \tsc{3m}-\tsc{tr}-do/say-\tsc{cont} there and then \tsc{3m}-be.located-\tsc{cont}=\tsc{rel} \tsc{3min} the.very.one crocodile\\
\ft There's a crocodile that climbs up from there, a single crocodile that lives there.
\exy \uex{Garrjarlngan inminjal booroo, booloomannganbard inminjal booroo, gardo boorroonganbard.}
\gll Garrjarl-ngan i-n-minjal booroo, boolooman-ngan=barda i-n-minjal booroo, gardo boorroo-ngan=barda.\\
frog-\tsc{all} \tsc{3m}-\tsc{tr}-wait.for place bullock-\tsc{all}=maybe \tsc{3m}-\tsc{tr}-wait.for place still kangaroo-\tsc{all}=maybe\\
\ft He might be looking for frogs, for bullock meat, or for kangaroo.
\exy \uex{Bardambmin ininirr  noonyjoo, inkalirr.}
\gll Bard=amb=min i-ni-n=irr noonyjoo, i-n-kal=irr.\\
away=\tsc{rel}=when \tsc{3m}-be.located-\tsc{cont}=\tsc{emph?} alive \tsc{3m}-\tsc{tr}-live=\tsc{emph?}\\
\ft He stays alive in there.
%\newpage
\exy \uex{Barnmin lardab indan  ginyingg oola, bardagid yardab indan  anyjamardan gaarrangan.}
\gll Barn=min larda=b i-n-joo-n ginyingg oola, barda=gid yardab i-n-d-an anyja-mardan gaarra-ngan.\\
that.way=when down=\tsc{rel} \tsc{3m}-\tsc{tr}-do/say-\tsc{cont} \tsc{3min} water away=\tsc{temp} crawl \tsc{3m}-\tsc{tr}-do/say-\tsc{cont} give.away-\tsc{dir} salt.water-\tsc{all}\\
\ft When the water goes down, then he crawls back to the ocean.
\exy \uex{Nyoonoo, joobool indan  gaarragoon.}
\gll Nyoonoo, joobool i-n-d-an gaarra-goon.\\
there swim \tsc{3m}-\tsc{tr}-do/say-\tsc{cont} salt.water-\tsc{loc}\\
\ft Now he's swimming in the sea.
\exy \uex{Biilagid aralgab inarn  oola bilarr daab indan  niindoo, ginyingginarda linygoorr, gard ararda, arramb arrmoonggoon.}
\gll Biila=gid aralga=b i-n-ar-n oola bilarr daab i-n-d-an niindoo, ginyinggin=arda linygoorr, gardi ar=arda, arr=amb arr-moonggoon.\\
again=\tsc{temp} next.day=\tsc{rel} \tsc{3m}-\tsc{tr}-pierce-\tsc{cont} rain swamp climb.up \tsc{3m}-\tsc{tr}-do/say-\tsc{cont} maybe that.same.one=maybe crocodile yet another=maybe \tsc{neg}=\tsc{rel} \tsc{1aug}-know\\
\ft When it gets wet again he climbs back up, I guess, maybe the same crocodile, maybe a different one, we don't know.
\exy \uex{Iyi, barnkarda gala.}
\gll Iyi, barnkard gala.\\
yes enough now\\
\ft That's the end.
\end{exye}

\subsection{AYI1: Fish traps (\emph{ayin})}
Jessie Sampi told this story to Claire Bowern in 2001 as part of a series of vernacular definitions (that is, short texts in Bardi which define and describe a particular word). It's a story about how Bardi people used to use \mor{ayin} stone fish traps. \mor{Ayin} are one of several types of traps used by Bardi people.

%\newpage
\setcounter{exxy}{0}\begin{exye}
\exy \uex{Ayin jina jawal nganjilnganjoogarr.}
\gll Ayin jina jawal nga-n-jilngi-n=joogarra.\\
fish.trap \tsc{3m.poss} story \tsc{1m}-\tsc{tr}-tell-\tsc{cont}=\tsc{2a.IO}\\
\ft I am telling you all a story about fish traps.
\newpage\exy \uex{Jamoonim jard inanggananamoord  ayinngan.}
\gll Jamoo-nim jard i-na-ng-ga-na-na=moord ayin-ngan.\\
maternal.grandfather-\tsc{erg} \tsc{1a.poss} \tsc{3m}-\tsc{tr}-\tsc{pst}-take-\tsc{rem.pst}-\tsc{rem.pst}=\tsc{1a.DO} fish.trap-\tsc{all}\\
\ft Our grandfather used to take us (to catch fish in) fish traps.
\exy \uex{Aralga boordaboorda inamanana  ayin gooljoo, goolboogid inamananajin.}
\gll Aralga boorda i-na-ma-na-na ayin gooljoo, goolboo=gid i-na-ma-na-na=jin.\\
next.day prepare \tsc{3m}-\tsc{tr}-put-\tsc{cont}-\tsc{rem.pst} fish.trap grass rock=\tsc{temp} \tsc{3m}-\tsc{tr}-put-\tsc{cont}-\tsc{rem.pst}=\tsc{3m.IO}\\
\ft He used to fix the fish trap. He added spinifex grass and he straightened the rocks (he weighted the grass with rocks so that the tide doesn't take it away).
\nt BE added the clitic \mor{=gid} when working through the text, to make it clear that the sense is that they weighted the grass with rocks so that the tide doesn't take it away.
\exy \uex{Jagoordgid injoonana  baalingan barda.}
\gll Jagoord=gid i-n-joo-na-n baali-ngan barda.\\
return=\tsc{temp} \tsc{3m}-\tsc{tr}-do/say-\tsc{rem.pst}-\tsc{cont} camp-\tsc{all} off\\
\ft Then he used to go back home.
\exy \uex{Mooyoonkid aralga, ``Way!,'' injoonanajard.}
\gll Mooyoon=gid aralga, ``Way!,'' i-n-joo-na-na=jard.\\
morning=\tsc{temp} next.day Come.on! \tsc{3m}-\tsc{tr}-do/say-\tsc{cont}-\tsc{rem.pst}=\tsc{1a.IO}\\
\ft Next morning he would say to us, ``Come on! Let's go fishing!''
%\nt BE 2008: better to say Aralga mooyoonkig way injoonanajard. the next morning he'd tell us to get up and go.
%\newpage
\exy \uex{Jooboolgid angirrinana  gaarragoon arrmarlang doombooldoombool angirrinana,  aarligid arinyjalboora ingarranana.}
\gll Joobool=gid a-ng-irr-i-na-na gaarra-goon arr-marla-ng doombool-doombool a-ng-irr-i-na-na, aarli=gid arinyjalboora i-ng-arr-a-na-na.\\
splash=\tsc{temp} 1-\tsc{pst}-\tsc{1aug}-do/say-\tsc{cont}-\tsc{rem.pst} salt.water-\tsc{loc} \tsc{1aug}-hand-\tsc{ins} slap.water-\tsc{redup} 1-\tsc{pst}-\tsc{1aug}-do/say-\tsc{cont}-\tsc{rem.pst} fish=\tsc{temp} one.place \tsc{3m}-\tsc{pst}-\tsc{1aug}-be.located-\tsc{rem.pst}\\
\ft We used to jump in the water and splash it with our arms so that the fish stayed in one place.
\nt The locative marker \mor{-goon} was added by BE in 2008, and she removed \mor{=gid} after \mor{gaarra}.
\exy \uex{Angirriminjalanana  gaarra inyjoodinanamarr,  niimanangarr aarli ingarrgardinana  ayinkoon boogoon.}
\gll A-ng-arr-minjala-na-na gaarra i-ny-joodi-na-na-marr, niiman=angarr aarli i-ng-arr-gardi-na-na ayin-koon boogoon.\\
1-\tsc{pst}-\tsc{1aug}-wait.for-\tsc{cont}-\tsc{rem.pst} tide \tsc{3m}-\tsc{pst}-dry.up-\tsc{cont}-\tsc{rem.pst}-sembl plenty=absolutely fish \tsc{3m}-\tsc{pst}-\tsc{1aug}-go.inside-\tsc{cont}-\tsc{rem.pst} fish.trap-\tsc{loc} inside\\
\ft We used to wait for the tide to go out, and (usually) lots of fish had gone inside the fish trap.
\exy \uex{Ginyinggon ongorronyinanirr.}
\gll Ginyinggon o-ng-orr-onyi-na-n=irr.\\
then 1-\tsc{pst}-\tsc{1aug}-kill-\tsc{cont}-\tsc{rem.pst}=\tsc{3a.DO}\\
\ft Then we killed them.
\exy \uex{Bardagid angarrananirr  niimanangarr aarli baalingan.}
\gll Barda=gid a-ng-arr-a-na-n=irr niiman=angarr aarli baali-ngan.\\
away=\tsc{temp} 1-\tsc{pst}-\tsc{1aug}-take-\tsc{cont}-\tsc{rem.pst}=\tsc{3a.DO} plenty=absolutely fish camp-\tsc{all}\\
\ft We used to go home with lots of fish.
\exy \uex{Angarramarrananagidirr,   angarrarlinanirr,   moorrgardamb daag \newline angirrinan.}
\gll A-ng-arr-a-marra-na-na=gid=irr, a-ng-arr-a-rli-na-n=irr, moorrgard=jamb daag a-ng-irr-i-na-n.\\
1-\tsc{pst}-\tsc{aug}-\tsc{tr}-cook:.it.cooked-\tsc{cont}-\tsc{rem.pst}=\tsc{temp}=\tsc{3a.DO} 1-\tsc{pst}-\tsc{aug}-\tsc{tr}-eat-\tsc{cont}-\tsc{rem.pst}=\tsc{3a.DO} feel.full=\tsc{rel} sleep 1-\tsc{pst}-\tsc{aug}-do/say-\tsc{cont}-\tsc{rem.pst}\\
\ft We used to cook them, and we used to go to sleep with a full stomach.
%\nt BE added 'eat' for sense.
\end{exye}

\subsection{COF1: Story about a tree coffin}
Nancy Isaac told this story to Claire Bowern in 2001. It's a story that her father used to tell about when he was a little boy.\index{Isaac, Nancy}

\setcounter{exxy}{0}\begin{exye}
\exy \uex{Aamb inyjiibina.}
\gll Aamba i-ny-jiibi-na.\\
man \tsc{3m}-\tsc{pst}-die-\tsc{rem.pst}\\
\ft A man had died.
\exy \uex{Nyoonamb ingarraman  garndi, jarrgandinygoon.}
\gll Nyoon=amb i-ng-arr-ma-n garndi, jarrgandiny-goon.\\
there=\tsc{rel} \tsc{3m}-\tsc{pst}-\tsc{1aug}-put-\tsc{rem.pst} top tree.coffin.platform-\tsc{loc}\\
\ft So they put him up high, in a tree coffin.
\exy \uex{Ingananagid  nganyjirrgoordal aalga, gooyarra irrjara aalga.}
\gll I-nga-na-na=gid nganyjigoord-al aalga, gooyarra irrjara aalga.\\
\tsc{3m}-\tsc{1m}-be.located-\tsc{rem.pst}=\tsc{temp} how.many-\tsc{indef} day two three day\\
\ft He was there I don't know how many days, maybe two or three days.
%\newpage
\exy \uex{Aralga barn ingirrinijin,  ``joo anggarrayarr  barda,'' ingirrinijin  ngajana gooloo.}
\gll Aralga barn i-ng-irr-i-ni=jin, ``joo a-ngg-arr-a-ya=rr barda,'' i-ng-irr-i-ni=jin ngajana gooloo.\\
another.day tell \tsc{3m}-\tsc{pst}-\tsc{aug}-do/say-\tsc{rem.pst}=\tsc{3m.IO} \tsc{2m} 1-\tsc{fut}-\tsc{1aug}-take-\tsc{fut}=\tsc{2m.DO} off \tsc{3m}-\tsc{pst}-\tsc{aug}-do/say-\tsc{rem.pst}=\tsc{3m.IO} \tsc{1m.poss} father\\
\ft One day they told him (my father), ``We'll take you somewhere,'' they told my father.
\exy \uex{``Joo anggarriyarri  bard arrjamb anja bard anjala ngaarr aamba \newline  nyoonoo joomboongan.}
\gll ``Joo a-ngg-arr-i-ya=rri barda arr=jamb anja bard a-n-jala ngaarri aamba nyoonoo j{oo}mboo-ngan.\\
\tsc{2m} 1-\tsc{fut}-\tsc{1aug}-take-\tsc{fut}=\tsc{2m.DO} off go=\tsc{rel} away off \tsc{2m-tr}-see ghost man there bereaved.sibling-\tsc{all}\\
\ft ``We'll take you. You walk first and go ahead by yourself. You want to see the dead man.''
\exy \uex{Roowil annya ngoordingan.}
\gll Roowil a-n-nya ngoordingan.\\
walk \tsc{2-tr}-catch alone\\
\ft ``Walk off alone.''
\exy \uex{Barda roowilroowil ingirrinyan,  nyoon ingarrjalgin,  baybirr marlmarl ingirrin  aamba.}
\gll Barda roowil-roowil i-ng-arr-inya-n, nyoonoo i-ng-arr-jalgi-n, baybirr marl-marl i-ng-irr-i-n aamba.\\
away walk-walk \tsc{3m}-\tsc{pst}-\tsc{aug}-catch-\tsc{cont} there \tsc{3m}-\tsc{pst}-\tsc{1aug}-hide-\tsc{cont} behind be.quiet-stop \tsc{3m}-\tsc{pst}-\tsc{aug}-do/say-\tsc{cont} man\\
\ft They men walked and they hid, they sat down quietly.
\exy \uex{Ngajana gooloo bard roowil innyana  gala ngoordingan.}
\gll Ngajana gooloo barda roowil i-n-nya-na gala ngoordingan.\\
\tsc{1m.poss} father off walk \tsc{3m}-\tsc{tr}-catch-\tsc{rem.pst} now alone\\
\ft My father walked off alone.
\exy \uex{Roowil innyana  o, jirrjirr injoonana  biila inyjarginan.}
\gll Roowil i-n-nya-na o, jirrjirr i-n-joo-na-na biila i-ny-jargi-na-n.\\
walk \tsc{3m}-\tsc{tr}-catch-\tsc{rem.pst} oh stand \tsc{3m}-\tsc{tr}-do-\tsc{cont}-\tsc{rem.pst} also \tsc{3m}-\tsc{pst}-be.afraid-\tsc{cont}-\tsc{rem.pst}\\
\ft He walked for a while, then he held back, standing there, getting frightened.
\exy \uex{Ginyinggon roowil innyana,  ``anggi angan gala ngalarga?''}
\gll Ginyinggon roowil i-n-nya-na, ``anggi angan gala nga-l-arg-a?''\\
then walk \tsc{3m}-\tsc{tr}-pick.up-\tsc{rem.pst} why why now \tsc{1m}-\tsc{irr}-be.afraid.of-\tsc{fut}\\
\ft He kept on walking. ``Why should I get frightened?'' (he asked himself).
\exy \uex{Roowil innyana  lagal ingganyin garndi.}
\gll Roowil i-n-nya-na lagal i-ng-ganyi-n garndi.\\
walk \tsc{3m}-\tsc{tr}-catch-\tsc{rem.pst} climb \tsc{3m}-\tsc{pst}-climb-\tsc{cont} top\\
\ft He kept on walking and climbed up on top.
\exy \uex{Nyoon inganana  garndi jarrgandinygoon aamba, ginyingg jarr \newline inganana,  ilogo injalanan.}
\gll Nyoonoo i-ng-na-na garndi jarrgandiny-goon aamba, ginyinggi jarri i-ng-na-na, ilogo i-n-jala-na-n.\\
there \tsc{3m}-\tsc{pst}-be.located-\tsc{rem.pst} top tree.coffin.platform-\tsc{loc} man \tsc{3min} this \tsc{3m}-\tsc{pst}-be.located-\tsc{rem.pst} side \tsc{3m}-\tsc{tr}-look.at-\tsc{cont-rem.pst}\\
\ft There was the man high in the tree coffin, he saw him there, lying on his side.
\exy \uex{O: nyalab boor ingarralana,  arangnim aamba nyalab darral inga-rrana  baybirrony, roowil ingirrinyan,  darr ingarranajin.}
\gll O nyalab boor i-ng-arr-jala-na, arang-nim aamba nyalab darral i-ng-arr-a-na baybirrony, roowil i-ng-irr-inya-n, darr i-ng-arr-a-na=jin.\\
oh over.there look.around \tsc{3m}-\tsc{pst}-\tsc{aug}-look.at-\tsc{rem.pst} others-\tsc{erg} man over.there come.out \tsc{3m}-\tsc{pst}-\tsc{aug}-take-\tsc{rem.pst} after walk \tsc{3m}-\tsc{pst}-\tsc{aug}-catch-\tsc{cont} come \tsc{3m}-\tsc{pst}-\tsc{aug}-bring-\tsc{rem.pst}=\tsc{3m.IO}\\
\ft They looked around, those other men who were coming along behind, they were walking, coming toward him (to check if he was frightened or not).
\exy \uex{``Gala yoorr anama balab,'' ingirrinijin.}
\gll ``Gala yoorr a-na-ma balab,'' i-ng-irr-i-na=jin.\\
now come.down \tsc{2-tr}-\tsc{tr}-put here \tsc{3m}-\tsc{pst}-\tsc{aug}-do/say-\tsc{rem.pst}=\tsc{3m.IO}\\
\ft ``Hey, you get down here!'' they said.
\exy \uex{Yoorr inaman balab.}
\gll Yoorr i-na-ma-n balab.\\
come.down \tsc{3m}-\tsc{tr}-put-\tsc{cont} here\\
\ft He came down.
\exy \uex{Ginyinggon biij o, ingoorroomoonggoorranan  ginyingg aamba, gaanyji-min injoonan  bardoon boonyja loogal injoonanajin.}
\gll Ginyinggon biij o, i-ng-oorr-moonggoorra-na-n ginyinggi aamba, gaanyji=min i-n-joo-na-n bardoon boonyja loogal i-n-joo-na-na jina.\\
then there oh \tsc{3m}-\tsc{pst}-\tsc{aug}-hold.up-\tsc{cont}-\tsc{rem.pst} \tsc{3min} man skeleton=when \tsc{3m}-\tsc{tr}-be.alive-\tsc{rem.pst}-\tsc{cont} skin all old \tsc{3m}-\tsc{tr}-do/say-\tsc{cont}-\tsc{rem.pst} \tsc{3m.poss}\\
\ft They used to wait until the man's flesh rotted and he became just skin and bone.
%\newpage
\exy \uex{Ingirrinyananagid  jin gaanyji ingarramananamba   baarlgoon.}
\gll I-ng-irr-inya-na-na=gid jina gaanyji i-ng-arr-a-ma-na-n=amb baarl-goon.\\
\tsc{3m}-\tsc{pst}-\tsc{aug}-pick.up-\tsc{cont}-\tsc{rem.pst}=\tsc{temp} \tsc{3m.poss} bone \tsc{3m}-\tsc{pst}-\tsc{aug}-\tsc{tr}-put-\tsc{cont}-\tsc{rem.pst}=\tsc{rel} paperbark-\tsc{loc}\\
\ft They would pick up his bones and then put them in paperbark.
\exy \uex{Ingorrorrgondondin  baarlgoon.}
\gll I-ng-arr-barrganda-n baarl-goon.\\
\tsc{3m}-\tsc{pst}-\tsc{aug}-tie.around-\tsc{cont} paperbark-\tsc{loc}\\
\ft They'd tie them up in the paperbark.
\newpage\exy \uex{Ingarrananamba, ingorrondinan  jardalngan agal booroo jirron ingarralabanan  jin gaanyji ginyingg aamba.}
\gll I-ng-arr-ar-na-n=jamb, i-ng-gondi-na-na jardal-ngan agal booroo jirr-on i-ng-arra-laba-na-n jina gaanyji ginyinggi aamba.\\
3-\tsc{pst-aug-}perce-\tsc{cont-rem.pst}=\tsc{thus} \tsc{3m}-\tsc{pst}-tie.up-\tsc{cont}-\tsc{rem.pst} bone.pillow-\tsc{all} and place \tsc{3a.poss}-\tsc{loc} \tsc{3m}-\tsc{pst}-\tsc{aug}-have-\tsc{cont}-\tsc{rem.pst} \tsc{3m.poss} bone \tsc{3min} man\\
\ft They'd tie it up and use it for a bone pillow and they'd keep the bones of the people in their houses.
\end{exye}

\subsection{ARL2: Story about catching fish with \emph{ilngam} fish poison}
David Wiggan told this story to Gedda Aklif in 1990. It's a short story about how to kill fish with \mor{ilngam} fish poison, one of the two plants used to stun fish so that they can be easily speared.%\enlargethispage*{1cm}

\setcounter{exxy}{0}\begin{exye}
\exy \uex{Aarli arralan barrbal arroongoorribinirr  janbal arranirr.}
\gll Aarli a-rr-ala-n barrbal a-rr-ngoorribi-n=irr janbal a-rr-a-n=irr.\\
fish 1-\tsc{aug}-live-\tsc{cont} golden.lined.spinefoot 1-\tsc{aug}-run.away-\tsc{cont}=\tsc{3a.DO} round.up 1-\tsc{aug}-[give]-\tsc{cont}=\tsc{3a.DO}\\
\ft When we see \mor{barrbal} fish, we chase them. We round them up. 
\exy \uex{Gaarra arrangajiman  irrolong, loolool irrgardin aarli bangalon.}
\gll Gaarra a-rr-ngajim-n irrol-ng, lool-ool irr-gardi-n aarli bangal-on.\\
sea 1-\tsc{aug}-kill.by.hitting-\tsc{cont} spear-\tsc{ins} enter-\tsc{redup} \tsc{3a-}go.inside-\tsc{cont} fish reef.crevice-\tsc{loc}\\
\ft We beat the sea water with spears and they go through the gap.
\newpage\exy \uex{Ginyinggon arriminjal gaarra marnanyib iyoodin, ginyinggon orronirr  irrolong aarli.}
\gll Ginyinggon a-rr-minjal gaarra marnany=b i-yoodi-n, ginyinggon o-rr-o-n=irr irrol-ng aarli.\\
then 1-\tsc{aug}-wait.for tide reef=\tsc{rel} \tsc{3m}-dry.up-\tsc{cont} then 1-\tsc{aug}-spear-\tsc{cont}=\tsc{3a.DO} spear-\tsc{ins} fish\\
\ft Then we wait for the tide to leave the reef dry and we spear the fish.
\exy \uex{Arra gorror jina irrola ilngamong orronyirr. Ginyinggon loogal irrin.}
\gll Arra gorror jina irrol ilngam-oong o-rr-ony=irr. ginyinggon loogal i-rr-i-n.\\
\tsc{neg} if \tsc{3m.poss} spear poison.root-\tsc{ins} 1-\tsc{aug}-kill=\tsc{3a.DO} then bad \tsc{3-aug-}do/say-\tsc{cont}\\
\ft If we don't have spears we poison them. Then they start to feel bad. 
\exy \uex{Gala aarli arinyjangarr arralan ginyinggon gala niimana goondoorr irralan.}
\gll Gala aarli arinyj a-rr-ala-n ginyinggon gala niimana goondoorr i-rr-ala-n.\\
now fish just.one 1-\tsc{aug}-live-\tsc{cont} then now very.many get.giddy.and.die \tsc{3m}-\tsc{aug}-live-\tsc{cont}\\
\ft First one, then many fish get dizzy.
\exy \uex{Ginyinggon gala boonyja goondoorr irralan.}
\gll Ginyinggon gala boonyja goondoorr i-rr-ala-n.\\
then now all get.giddy.and.die \tsc{3m}-\tsc{aug}-live-\tsc{cont}\\
\ft Then they all get dizzy and die.
%\newpage
\exy \uex{Irrjimbin ginyinggon arrooloonganirr,  ginyinggon jolonjolon arramanirr  irrolon.}
\gll I-rr-jiibi-n ginyinggon a-rr-ooloong-an=irr, ginyinggon jolonjolon a-rr-a-ma-n=irr irrol-on.\\
\tsc{3m}-\tsc{aug}-die-\tsc{cont} then 1-\tsc{aug}-collect-\tsc{cont}=\tsc{3a.DO} then string.on.spear 1-\tsc{aug-tr}-put-\tsc{cont}=\tsc{3a.DO} spear-\tsc{loc}\\
\ft They die, then we pick them up and we put them on our spears.
\newpage\exy \uex{Ooranynim oolordon irramanirr,  wiinyja irramanirr  oolarda.}
\gll Oorany-nim oolard-on i-rr-a-ma-n=irr, wiinyja i-rr-a-ma-n=irr \.{oo}larda.\\
woman-\tsc{erg} coolaman-\tsc{loc} \tsc{3m}-\tsc{aug}-\tsc{tr}-put-\tsc{cont}=\tsc{3a.DO} fill.up \tsc{3m}-\tsc{aug}-\tsc{tr}-put-\tsc{cont}=\tsc{3a.DO} coolaman\\
\ft The women put them in their baskets; they fill them up.
\exy \uex{Ginyinggon barda rawin arraman  baaliboorngan. Ginyinggon arrooloo-rroon noorroo.}
\gll Ginyinggon barda rawin a-rr-a-ma-n baaliboor-ngan. Ginyinggon a-rroo-loorroo-n noorroo.\\
then off go.as.group 1-\tsc{aug}-\tsc{tr}-put-\tsc{cont} camp-\tsc{all} then 1-\tsc{aug}-light-\tsc{cont} fire\\
\ft Then away we all go to the camp. Then we light a fire.
\exy \uex{Ginyinggon arramarranirr  aarli.}
\gll Ginyinggon a-rr-a-marra-n=irr aarli.\\
then 1-\tsc{aug}-\tsc{tr}-burn-\tsc{cont}=\tsc{3a.DO} fish\\
\ft Then we cook the fish.
\exy \uex{Boonyja imbanjanjinana  barninyarrang, arra barnanggarrmarr gardi jin aarli innyan  noordingan goorlil.}
\gll Boonyja i-m-banj-anji-na-na barninyarrang, arra barnanggarr-marr gardi jina aarli i-n-nya-na ngoordingan goorlil.\\
all \tsc{3m}-\tsc{pst}-share-\tsc{redup}-\tsc{cont}-\tsc{rem.pst} the.whole.thing \tsc{neg} today-\tsc{sembl} yet \tsc{3m.poss} fish \tsc{3m}-\tsc{tr}-catch-\tsc{rem.pst} alone turtle\\
\ft People used to share everything, but today they get fish and turtle for themselves.
\end{exye}

\section*{Further textual materials}
There are four books of Bardi narratives available to Bardi people. For reasons of cultural sensitivity, Bardi speakers have asked that the unpublished materials not be made publicly available in their entirety at this stage. The `blue book' \citep{bow02jnj} is a set of selected narratives recorded by Gedda Aklif and Claire Bowern as part of a Bardi oral history project. Some of those texts were also put in a smaller book \citep{jawal}. Some of the texts recorded by C.D. (Toby) Metcalfe from Tudor Ejai and Johnny Boxer appeared in \citet{hersut86}. Finally, a selection of narratives from Laves' materials was typed, translated, and annotated in 2003. 
